% ----------------------------------------------------------------
% Thesis - Main document
% ----------------------------------------------------------------

\documentclass{thesisclass}
% as defined in thesisclass.cls


%% ---------------------------------
%% | Information about the thesis  |
%% ---------------------------------

\newcommand{\myname}{Jingjian Wei}
\newcommand{\mytitle}{Adversarial Attacks and Defenses for Point Cloud based 3D Object Detection Network}

\newcommand{\reviewerone}{Prof. Dr.-Ing. Christoph Stiller}
\newcommand{\advisor}{Juncong Fei, M.Sc., Ahmed Hammam, M.Sc.}

\newcommand{\timeandplace}{Karlsruhe, Mai 2021}

\newcommand{\frontmatterhint}{}

%% -------------------------------
%% |  Information for PDF file   |
%% -------------------------------
\hypersetup{
 pdfauthor={\myname},
 pdftitle={\mytitle},
 pdfsubject={Thesis},
 pdfkeywords={}
}
\usepackage[english]{babel}  	% English
%\usepackage[ngerman]{babel}  	% German


%% -------------------------------
%% |    My packages /commands    |
%% -------------------------------
% track changes
    %\usepackage{changes} 		% highlight changes
	\usepackage[final]{changes}	% don't highlight changes

% tables
	\usepackage{array,multirow,graphicx}

% pseudocode
	\usepackage[ruled,vlined]{algorithm2e}

% for forcing positions of tables, figures, ...
	\usepackage[capposition=bottom]{floatrow}

% units
	\usepackage{siunitx}
	\usepackage{dcolumn}
%latex special character
    \usepackage{latexsym}
% math formular
\usepackage{indentfirst} 
\usepackage{mathtools}
\usepackage{commath}
\usepackage{amsmath}
%footnote
   % \usepackage[backend=biber,style=verbose-trad2]{biblatex}
% 4th section
\setcounter{secnumdepth}{4}
\setcounter{tocdepth}{4}
% \usepackage{titlesec}

% \setcounter{secnumdepth}{4}

% \titleformat{\paragraph}
% {\normalfont\normalsize\bfseries}{\theparagraph}{1em}{}
% \titlespacing*{\paragraph}
% {0pt}{3.25ex plus 1ex minus .2ex}{1.5ex plus .2ex}




% minus in formular
\usepackage{graphicx}
\newcommand{\minus}{\scalebox{0.75}[1.0]{$-$}}


%% --------------------------------
%% |      glossaries               |
%% --------------------------------
\newif\ifuseglossaries			% as the glossaries package sometimes causes problems, you can switch if of
								% if switching of, remember to remove all \gls{}/... statements
								
\useglossariestrue				% use glossaries (acronyms and online references)
%\useglossariesfalse			% don't use glossaries

\ifuseglossaries
	\usepackage[acronym,xindy,toc,nomain,nonumberlist,nopostdot,notranslate]{glossaries}%

	\newglossary[onlineref]{onlineref}{onli}{onlo}{Online References}
	
	\newglossarystyle{mylong}{%
  		\setglossarystyle{long}%
  		\renewcommand*{\glsnamefont}[1]{\textbf{##1}}%
  	}%
  	
  	\newglossarystyle{mylist}{%
  		\setglossarystyle{list}%
  		\renewcommand*{\glsnamefont}[1]{\textrm{\textmd{##1}}}%
  	}%
  	
  	\makeglossaries
  	
	% ordinary acronyms
	\newacronym{ma}{MA}{My Acronym}
	\newacronym{dnn}{DNN}{Deep Learning Neural Network}
	\newacronym{slam}{SLAM}{Simultaneous Localization and Mapping}
	\newacronym{cnn}{CNN}{Convolutional Neural Network}
	\newacronym{rcnn}{R-CNN}{Region Based Convolutional Neural Network}
	\newacronym{yolo}{YOLO}{You Only Look Once}
	\newacronym{lidar}{LiDAR}{Light Detection and Ranging}
	\newacronym{fgsm}{FGSM}{Fast Gradient Sign Method}
	\newacronym{pgd}{PGD}{Projected Gradient Descent}
	\newacronym{svm}{SVM}{Support Vector Machine}
	\newacronym{cw}{C\&W}{The Carlini and Wagner Method}
	\newacronym{jsma}{JSMA}{Jacobian-based Saliency Map Attack}
	\newacronym{roi}{RoI}{Region of Interest}
	\newacronym{rpn}{RPN}{Region Proposal Network}
	\newacronym{fpn}{FPN}{Feature Pyramid Network}
	\newacronym{ssd}{SSD}{Single Shot MultiBox Detector}
	\newacronym{ap}{AP}{Average Precision}
	\newacronym{coco}{COCO}{Microsoft Common Objects in Context}
	\newacronym{mlp}{MLP}{Multi-Layer Perceptron}
	\newacronym{bev}{BEV}{Bird’s Eye View}
	\newacronym{dl}{DL}{Deep Learning}
	\newacronym{vfe}{VFE}{Voxel Feature Encoding}
	\newacronym{iou}{IoU}{Intersection Over union}
	\newacronym{voc}{VOC}{Visual Object Classes}
	\newacronym{second}{SECOND}{Sparsely Embedded Convolutional Detection}
	\newacronym{ifgsm}{IFGSM}{Iterative Fast Gradient Sign Method}
	\newacronym{mifgsm}{MIFGSM}{Momentum Iterative Fast Gradient Sign Method}
	\newacronym{aos}{AOS}{Average Orientation Similarity}
	\newacronym{fps}{FPS}{Furthest Point Sampling}

	% usage: \gls{} \glspl{} or \acrshort{}
	
	% online references
	\newglossaryentry{online_reference}{
		type=onlineref, 
		name={[01]}, 
		description={\hspace{1.5mm}My Online Reference\newline 
			\href{http://www.mrt.kit.edu/}{http://www.mrt.kit.edu/}\newline 
			retrieved 2016-04-28 }}
\else
\fi


%% ---------------------------------
%% | ToDo Marker - only for draft! |
%% ---------------------------------
% Remove this section for final version!
\setlength{\marginparwidth}{20mm}

\newcommand{\margtodo}
{\marginpar{\textbf{\textcolor{blue}{ToDo1}}}{}}


\newcommand{\margmajortodo}
{\marginpar{\textbf{\textcolor{red}{ToDo2}}}{}}

\newcommand{\majortodo}[1]
{{\textbf{\textcolor{red}{(\margmajortodo{}#1)}}}{}}

%% --------------------------------
%% | Draft Marker - only for draft! |
%% --------------------------------
% Remove this section for final version!
%\usepackage{draftwatermark}
%\SetWatermarkText{\hspace{10pt}-draft-}
%\SetWatermarkScale{2.3}
%\SetWatermarkColor[gray]{0.95}


%% --------------------------------
%% | Old Marker - only for draft! |
%% --------------------------------
% Remove this section for final version!
\newenvironment{deprecated}
{\begin{color}{gray}}
{\end{color}}


%% --------------------------------
%% | Settings for word separation |
%% --------------------------------
% Help for separation:
% In german package the following hints are additionally available:
% "- = Additional separation
% "| = Suppress ligation and possible separation (e.g. Schaf"|fell)
% "~ = Hyphenation without separation (e.g. bergauf und "~ab)
% "= = Hyphenation with separation before and after
% "" = Separation without a hyphenation (e.g. und/""oder)

% Describe separation hints here:
\hyphenation{
Wort-tren-nung
% Ma-na-ge-ment  Netz-werk-ele-men-ten
% Netz-werk Netz-werk-re-ser-vie-rung
% Netz-werk-adap-ter Fein-ju-stier-ung
% Da-ten-strom-spe-zi-fi-ka-tion Pa-ket-rumpf
% Kon-troll-in-stanz
}

%%%%%%%%%%%%%%%%%%%%%%%%%%%%%%%%%
%% Here, main documents begins %%
%%%%%%%%%%%%%%%%%%%%%%%%%%%%%%%%%
\begin{document}


\frontmatter

\include{Preface/01_Title}
\blankpage
\include{Preface/02_Declaration}
\blankpage
\include{Preface/03_Acknowledgement}
\blankpage
% !TeX root = ../thesis.tex

\chapter*{Abstract}
\addcontentsline{toc}{chapter}{Abstract}





\chapter*{Kurzfassung}
\addcontentsline{toc}{chapter}{Kurzfassung}

\blindtext

\blankpage

\tableofcontents

% Acronyms
\ifuseglossaries
	\newpage
	\thispagestyle{empty}
	\printglossary[type=acronym, style=mylong]
\else
\fi


%% -----------------
%% |   Main part   |
%% -----------------
\mainmatter
\pagenumbering{arabic}
% !TeX root = ../thesis.tex

\chapter{Introduction}
\label{sec:introduction}
Within the last few years, substantial research in deep learning networks for object detection have enabled the autonomous vehicles to develop rapidly and achieve promising performance. Rather than simply detecting the road the vehicle is driving on, the surrounding people and vehicles, how they act and how to respond are also crucial in determining the possible future trajectories and realizing the safe driving. The ability to sense the surrounding environment is called perception, including multiple subdomains such as objection detection\cite{girshick_rich_2014}, classification\cite{krizhevsky_imagenet_2017}, simultaneous localization and mapping (\acrshort{slam})\cite{smith_estimating_2013,montemerlo_fastslam_nodate}. Astounding advances in computer vision have been achieved in the last decade and undoubtedly its ability can rival or even surpass human in many areas. Before the emergence of deep learning, a time-consuming method “the sliding window”\cite{enzweiler_monocular_2009} is used to detect objects, in which a rectangle window moves over the image to find the target object and the classifier has to be applied in each window to classify the interested image. Methods relying on neutral network and the advance in reliable perception systems facilitate the tasks of object detection to be more efficient and accurate. The invention of \acrshort{cnn}\cite{lecun_backpropagation_1989} made the milestone contribution to deep learning algorithms and the application in computer vision, such as \acrshort{rcnn}\cite{girshick_rich_2014},Fast R-CNN\cite{girshick_fast_2015}, \acrshort{yolo}\cite{redmon_you_2016}, consecutively improvement in which achieving a progressively performance in object detection work.

Considering onboard sensors used to assess the surroundings, cameras have been well accepted for many years due to the price advantage and relatively clear recognition. However, 2D image is not qualified to provide deeply scene perception required in automotive and industrial robotics, while the astonishing development in 3D sensors, such as ToF (time-of-flight), \acrshort{lidar}, stereo vision etc., light up the future of object detection. \acrshort{lidar}, short for Light Detection and Ranging, is equipped in many autonomous vehicles to capture 3D scene information with the representation of sparse and irregular point clouds. Comparing with the conventional camera, \acrshort{lidar} sensor is capable of providing high-resolution \(360^{\circ}\) 3D information, as shown in Figure \(\ref{fig:lidar scan range}\), by discretizing the vertical space in line with hundreds of points.  In the object detection work, the raw point clouds generated by \acrshort{lidar} sensor will go through a preprocess firstly, and then be fed into a detection model, the output of which will be post-processed to make predictions. The advantages of \acrshort{lidar}, such as accurate estimation of object sizes when many similar colored objects exist and less affected by bad lighting via providing own infrared light source, make it considerably competitive in autonomous driving, especially under adverse weather conditions.

\begin{figure}[!htbp]
\centering
\includegraphics[scale=0.6]{Graphics/Lidar scan range.jpg}
\caption{The horizontal scan range of \acrshort{lidar} and camera\cite{Lidar}}
\label{fig:lidar scan range}
\end{figure}

The high reliability of the point cloud information makes \acrshort{lidar}-based perception become more and more welcomed and adopted in autonomous vehicles industry, but inevitably \acrshort{lidar} has its limitation, for instance, the difficulty of detection exponentially increases with the distance from the object. Moreover, recent studies point out that occlusion patterns in \acrshort{lidar} point clouds can be ignored and severely threaten the credibility of detection\cite{sun_towards_nodate}.  Specifically, when one vehicle is driving behind another one, the point cloud displays fewer points because the front vehicle occludes the \acrshort{lidar} beams. Inability to learn occlusion information makes \acrshort{lidar} point clouds expose to spoofing attacks, which encourage further exploration in the underlying misclassification of object detection imperatively.

Many studies have shown that small perturbations in the input sample, which are even imperceptible to human vision, can prompt the model to make a substantially incorrect prediction with high confidence\cite{papernot_limitations_2015,szegedy_intriguing_2014},\cite{goodfellow_explaining_2015} possibly leading to devastating consequences in the real world such as car crashes and fatal injuries\cite{kurakin_adversarial_2017,Uber}. This vulnerability of neural network restricts its application in security-critical area and make model robustness test under different scenarios increasingly needed. “Adversarial examples” was firstly introduced by Szegedy et al.\cite{szegedy_intriguing_2014} in the image classification work,  generating the perturbed examples by exploring the minimal necessary perturbations to maximize the prediction error, and adversarial examples have transferability property which can be used to attack on diverse models\cite{szegedy_intriguing_2014,goodfellow_explaining_2015,liu_delving_2017,papernot_transferability_2016,naseer_cross-domain_2019}. After the revelation of the existence of adversarial examples,  multitude researchers dedicate to inspecting how robust a neural network can be and how to strengthen networks against the attacks\cite{huang_learning_2016,gu_towards_2015,bastani_measuring_2017,shaham_understanding_2018}, activating a fresh study domain “adversarial attack and defense”.  

Concerning the categories of threat model, adversarial attacks can be classified into white box, grey box and black box, depending on the degree of knowledge acquired by the adversary. White-box attack assumes the adversary has comprehensive knowledge of the target model, therefore, crafting adversarial samples on the target model is relatively simple and direct, and most well-known attack algorithms are designed based on white-box model. The knowledge of the adversary in the grey-box attacks is limited, with only the information of target model structure\cite{ferrari_gray-box_2018}. In the black-box attacks, the attacker has no access to the target model parameters\cite{sun_towards_nodate,liu_delving_2017,papernot_practical_2017}.  Innovative and progressive attack algorithms come out continuously during the last couple of years, such as \acrshort{fgsm}\cite{goodfellow_explaining_2015}, \acrshort{pgd}\cite{kurakin_adversarial_2017}, \acrshort{cw}\cite{carlini_towards_2017}, \acrshort{jsma}\cite{papernot_limitations_2015}, DeepFool\cite{moosavi-dezfooli_deepfool_2016}. 

To increase the network robustness against above enumerated attacks, defense approaches are urgently required, which generally includes adversarial training\cite{szegedy_intriguing_2014}, denoising methods\cite{zhou_dup-net_2019,xie_feature_2019,xu_feature_2018}, randomization-based schemes\cite{luo_random_2020,athalye_synthesizing_2018}, provable defenses\cite{raghunathan_certified_2020}, and detection of adversarial examples before feeding into the networks\cite{meng_magnet_2017,liu_detection_2018}. Amid the current defense mechanisms, none of them can manifest both efficient and effective against existing adversarial samples, all bearing either deficiency of computational intractability or fragility to adaptive white-box attacks to some extent. Conceivable explanations accounting for the obstacles in constructing defense mechanisms are as following: lack of powerful theoretical tools to solve complicated optimization problems which are commonly used to generate adversarial samples,  impracticable of deep learning to produce good outputs for each possible input. Defenses techniques with high adaptivity and effectiveness 

Adversarial attacks for 2D image data is well explored in recent research as introduced above\cite{szegedy_intriguing_2014,goodfellow_explaining_2015,moosavi-dezfooli_deepfool_2016,papernot_limitations_2015,carlini_towards_2017}, but studies on the vulnerability of point cloud network is not sufficient until now. Extending 2D adversarial attacks approaches to point clouds is a challenging task, considering the following reasons: 

1. Raw point clouds are situated in coordinates \(xyz\), without pixel information that can be modified. 

2. The positions of points are arbitrary which enlarge the search space for adversarial samples. 

3. \(L_{p}\) norm that is widely accepted in 2D adversarial attack is not applicable in point clouds data regarding to the properties of irregularity and varying cardinality\cite{xiang_generating_2019}. So far, 3D adversarial attack approach is composed of following categories: gradient-based method such as FGSM\cite{goodfellow_explaining_2015} and its variants\cite{gu_towards_2015,kurakin_adversarial_2017,dong_boosting_2018,madry_towards_2019,liu_extending_2019}, optimization-based method represented by C\&W\cite{carlini_towards_2017} and L-BFGS\cite{szegedy_intriguing_2014}, skeleton-detach based approach\cite{yang_adversarial_2019} and GAN-based method\cite{zhou_lg-gan_nodate,vedaldi_advpc_2020}. 

Heretofore, most attack approaches are proved to be theoretically feasible under preset scenarios and carefully-crafted adversarial sample datasets. Considering the potential applications in security-critical systems such as autonomous vehicles and the unpredictable nature of 3D data which can be affected by lighting or noise in adverse weather, comprehensive understanding of robustness on practical examples in the real world is of paramount importance. To the best of our knowledge, physically attacks on 3D point clouds are simulated in some recent works, for instance, setting obstacles under different distance and orientation\cite{cao_adversarial_2019}, placing adversarial objects on the rooftop of a vehicle\cite{tu_physically_2020}, simulating "hide objects" by randomly removing original points in ground-truth bounding box\cite{hau_object_2021} These attempts prove that existing state-of-the-art detection networks are disappointingly susceptible to attacks and pose highly concern on the practical applications. In spite of the poor results, these endeavours are very meaningful to call for and shed light on more future effort in 3D point cloud detection.

In this paper, we are aiming at exploring the robustness of 3D point cloud object dection networks. The experiment is composed of four sections: In the first section, we extend FGSM algorithm and its variants, which are widely used in image domain and point cloud classification area, to 3D point cloud objetct detection, and carry out attacks and defenses on the neural network, specifically PointPillars\cite{lang_pointpillars_2019}. In the second section, we perform adversarial attacks by the means of dropping critical points, which is inspired by Zheng et al.\cite{zheng_pointcloud_2019} PointPillars use a simplified PointNet\cite{qi_pointnet_2017} to capture critical features, and we define critical points according to the abstracted critical features. Attacks in the first two sections are based on detection models, a new adversarial attack approach not relying on a specific model is introduced in the section three. We create benchmark datasets by either adding Gaussian noise or randomly drop points to test the volunerability of the models. In the last section, we augment dataset by integrating simulated point clouds of adverse weather. 

The paper is constructed as follows: The first section describes the pertinent background and motivation , formalize the problem 
In the second section, related literature about object detection and adversarial attack and defenses are briefly introduced.  Section 3 will provide detailed description of the dataset and methodology. Section 4 will present the experiment settings and results of attacks and defense and analysis. The last section summarizes the conclusion and discuss the future directions.
      
% !TeX root = ../thesis.tex

\chapter{Related Works}
\label{sec:fundamentals_related-work}
In this section, we review the literature in neural network and introduce some methods of adversarial attacks.
\section{Image Object Detection Network}

Deep Neural Networks(\acrshort{dnn}) has shown highly expressive performance in the last decade and has become an indispensable tool in many practical tasks such as image classification, natural language processing and object detection. In 2012, Krizhevsky et al.\cite{krizhevsky_imagenet_2017} use a 5-layers convolutional neural network named as AlexNet achieving a considerably outstanding performance than their closest competitor \acrshort{svm}, ushering a new era in deep neural networks and encouraging further exploration in image classification. 

With the improvement of machine learning models and object representations, complete image understanding including precisely determine the class and location of the objects, known as object detection, becomes achievable and crucial. A detection work is to classify every sub-window in the image, which requires huge computational work. Two methods, region-proposal methods and end-to-end systems, are proposed to address the issues.
\subsection{Region Proposal Method}
Region-proposal method is a two-step process, taking a basic scan to suggest the promising locations and then evaluating the interest region with \acrshort{dnn} classifier. This method mainly include \acrshort{rcnn}\cite{girshick_rich_2014}, SPPNet\cite{he_spatial_2014}, Fast R-CNN\cite{girshick_fast_2015}, Faster-RCNN\cite{ren_faster_2016}, R-FCN\cite{dai_r-fcn_2016},\acrshort{fpn}\cite{lin_feature_2017}, and Mask R-CNN\cite{he_mask_2018}. 

R-CNN was proposed in 2014 by Girshick et al.\cite{girshick_rich_2014}. This network can be divided into three stages, extracting region proposals, capture features of each proposal, and classify regions with \acrshort{svm}. In the first stage, \acrshort{rcnn} takes selective search\cite{uijlings_selective_2013} to generate thousands region proposals for each image using bottom-up grouping and saliency score. Then, these chosen proposals are wrapped or cropped into a settled resolution and a 4096 dimensional feature vector is extracted by CNN as Krizhevsky proposed(ImageNet\cite{deng_imagenet_2009}). In the last step, each region proposal is scored with a trained class-specific linear \acrshort{svm}s. \acrshort{rcnn} has good performance and brings CNN into practical object detection, but both the searching and training procedures are time-consuming and storing features requires a large storage space.

Not only the speed needs to be improved, objects may also have incomplete or distort problems in the wrapping procedure. To solve these issues, He et al.\cite{he_spatial_2014} introduce spatial pyramid pooling(SPP) net, by reusing the feature maps based on the spatial positions to avoid repeated computation.

Inspired by the SPP-net, the developer of R-CNN Girshick improve his network by introducing a region of interest(\acrshort{roi}) pooling layer, which is a special case of SPP layer with only one pyramid layer\cite{girshick_fast_2015}. The training of all networks is optimized in one stage with multi-task loss on classification and bounding box regression. Fast R-CNN requires less storage space and improve the efficiency and accuracy further.

Ren et al\cite{ren_faster_2016} introduce Regional Proposal Network(\acrshort{rpn}) to replace the selective search methods. \acrshort{rpn} shares full-image convolutional layers with detection network and can predict object bounds and outputs score at each position. The proposal faster R-CNN signifies that the region proposal-based \acrshort{cnn} can be trained in end-to-end method, but the training algorithm is quite time-consuming.

Li et al.\cite{dai_r-fcn_2016} proposed region-based fully convolutional networks(R-FCN) to break down the translation invariance by inserting the \acrshort{roi} pooling layer into the convolutions manually. Compared with Faster R-CNN, the last layer of this methods outputs \(k^{2}\) position-sensitive score maps, being averaged to produce a (C+1) dimension vector, and obtain class-agnostic bounding box by adding another \(4k^{2}\) dimension convolutional layer. 

The architecture of FPN\cite{lin_feature_2017} is built with a bottom-up pathway, which downsamples the feature map and chooses the last layer output as the reference set of feature maps, a top-down pathway, where feature maps are upsampled and enhanced with same spatial size maps from bottom-up pathway, and several lateral connections. This method does not rely on CNN architectures, and can be used in object detection and other studies in computer vision, such as instance segmentation.

Instance segmentation includes two steps: detecting all the objects in the image and segmenting the instance, also known as semantic segmentation. Problems such as spurious edge or potential systematic errors on overlapping instances might occur\cite{arnab_pixelwise_2017}, and Mask R-CNN\cite{he_mask_2018} add a segmentation mask branch to solve the issues. Mask R-CNN adopts RoIAlign layer, which uses bilinear interpolation to compute the values of input features, replacing the spatial quantization for features extraction in 
\acrshort{roi} pooling. Although this improvement increases the computational volume, the substantial enhancement in accuracy and its flexibility to cooperate with other tasks make this instance-level recognition method accepted in objection detection.

\subsection{End-to-end Systems}
Different from the region proposal framework, which has multistage and each stage is trained separately, end-to-end system has only one stage using regression or classification to map image pixels straightly to bounding box. Two typical frameworks of this methods are \acrshort{yolo}\cite{redmon_you_2016} and \acrshort{ssd}\cite{liu_ssd_2016}.

Many researchers have tried to apply regression or classification to model object detection before \acrshort{yolo} and \acrshort{ssd}, but huge computing difficulty and inability to deal with overlapping objects obstruct the development. 

\acrshort{yolo}(You only look once) is a novel framework proposed by Redmon et al.\cite{redmon_you_2016}, which predict confidences and bounding boxes with the whole topmost feature map. \acrshort{yolo} consists of 24 convolutional layers and 2 fully connected layers, and it can learn priors on object positions. Additionally, decrease in false positives on backgrounds makes it possible for \acrshort{yolo} to cooperate with Fast R-CNN. Later YOLOv2, an upgrade version, adopted high resolution classifier, dimension clusters, anchor boxes etc., exhibiting expressive improvement in m\acrshort{ap} and real-time performance.\cite{redmon_yolo9000_2016}

Another end-to-end approach \acrshort{ssd}(Single Shot MultiBox Detector) was suggested by Liu et al. \cite{liu_ssd_2016}, which realizes a better balance between precision and speed. \acrshort{ssd} runs a convolutional network on the input image only once to extract a feature map, uses default anchor boxes at a variety of aspect ratios and scales, and predict bounding boxes from multiple feature maps with varied resolutions to deal with the object scale. \acrshort{ssd} realizes an outperformance than Faster R-CNN in accuracy and almost three times faster on PASCAL VOC\cite{everingham_pascal_2010} and \acrshort{coco}\cite{lin_microsoft_2015}, and more efficient than \acrshort{yolo} in dealing with large-size objects.

\section{PointCloud Classification}
3D data complement 2D images, providing shape and scale information to get a more thorough understanding of the surroundings. The representations of 3D data include point clouds, depth-maps, volumetric grids, meshes etc., and \acrshort{lidar} point clouds are used in this paper. 

Point clouds can be classified as Multiview-based, point-based and voxel-based methods. Multiview based methods firstly project point clouds into multiple views, extract and fuse features for classification. 

Su et al.\cite{su_multi-view_2015} introduce MVCNN, which is a pioneer work, proving that combining multiple views of a 3D shape information into a compact descriptor could attain a good recognition performance. The procedure of aggregating multi-view features to global shape descriptor is the main challenge of this approach, which inspires researchers to exploit progressive feature aggregation strategy. 

One attempt of Wei and his colleagues \cite{wei_view-gcn_2020} is to regard multiple views as graph nodes and construct a hierarchical graph \acrshort{cnn} to learn global shape descriptor. They name the network as view-GCN, which achieves high accuracy in point cloud classification and retrieval and outperform other methods on RGB-D dataset\cite{lai_large-scale_nodate}.

Voxel-based method, also known as volumetric-based method, is to voxelize a point cloud into 3D grids and classify the point cloud by introducing volumetric representation on CNN training. With the proposal of VoxNet\cite{maturana_voxnet_2015} by Maturana, the idea of transforming a 3D model into an occupancy grid is widely accepted and encourages promotion of the VoxNet and its variants. 

Wu et al\cite{wu_3d_2015} propose to use a binary probability distribution on a voxel grid to demonstrate a geometric 3D shape, applying on convolutional deep belief network. The recognition accuracy is impressive on various recognition tasks and on different datasets, but the ability to scale dense 3D data is limited due to the large computational and storage requirements. 

To better represent the point clouds geometry information, PointGrid was introduced by Le et al.\cite{le_pointgrid_2018}, which is \acrshort{cnn} integrated point and grid. The network is embedded a regular-structured volumetric grid to hierarchically extract global information and incorporates constant points number within each grid cell to overcome the limitation of grid cell. Compared with existing state-of-art methods, this approach has advantages in speed and memory requirements.
  
Methods based on multiple view images and voxel grids all adopt the views from image neural network, different from them, point-based approach works on raw point clouds directly, simplifying the process and reducing the information loss during projection or voxelization. In terms of network architecture, point-based method can be categorized into convolution-based, graph-based, pointwise \acrshort{mlp} and other methods. 

Considering the irregularity of point data, point-wise \acrshort{mlp} methods use Multi-Layer Perceptron to process each point independently, then use a symmetric aggregation function to derive a global feature. An innovative and simple network PointNet\cite{qi_pointnet_2017} directly takes point clouds as input and treat these input points independently and identically. Max pooling layer uses a single symmetric function, respecting the permutation invariance, to select informative points in the point cloud, which is the key part of PointNet. In the final step, the network aggregate learnt information to the global descriptor for further point cloud classification and segmentation. PointNet exhibits strong stability and efficiency empirically and the ideas are transferred and advanced in further research.

Incapability to capture local structures of PointNet motivates Qi et al.\cite{qiPointNetDeepHierarchical2017} to put forward an improved version, PointNet++. This hierarchical network partitions point sets into local regions according to the metric space distance firstly, then further group local features to capture high-level representations until collect the features of the whole point set. Unlike the single max pooling layer, PointNet++ involves a number of set abstraction levels, which is made of sampling layer, grouping layer and PointNet layer. By abstracting increasingly broader local regions along the hierarchy, PointNet++ is able to capture local features with increasingly larger scales, and the efficiency and robustness are both proved significantly better than other benchmarks in point clouds study.

\section{\acrshort{lidar}-based Object Detection Network}
The availability and reliability of 3D sensors such as \acrshort{lidar}, RGB-D camera and various 3D scanners, make it possible for autonomous vehicles equipped with multiple sensors collect more accurate and detailed information. The input of 3D object detector is the point clouds of a scene and a 3D bounding box is sketched around the detected target. The approaches of 3D object detection are divided into two types: region proposal methods and single shot methods. 
\subsection{Region Proposal-based Methods}
Adopting the idea in 2D image object detection, region proposal-based methods in 3D object detection consists of three categories regarding to the means of generating object proposal, specifically Multiview-based, segmentation-based and frustum-based, and the representative networks of which are shown in Figure \(\ref{fig:LiDAR Region Proposal}\)
\subsubsection{Multi-view based Methods}
The general idea of Multiview methods is integrating proposal-wise features from various point cloud projections, such as \acrshort{lidar} front view and \acrshort{bev} (Bird’s Eye View), to generate 3D rotated boxes. The specific process is shown in Figure\(\ref{fig:LiDAR Region Proposal}\) (a). MV3D, a sensory-fusion framework, is proposed by Chen et al.\cite{chen_multi-view_2017} Taking \acrshort{lidar} point cloud and RGB images as the input to generate oriented 3D bounding boxes. The network consists of two subnetworks, to generate 3D object proposals and to fuse multi-view features respectively. 3D candidate boxes are generated in the first subnetwork from the \acrshort{bev} map projected by 3D point clouds, and region-wise features from multiple views are combined in the deep fusion scheme in the second subnetwork. MV3D achieves higher AP in 3D localization and detection than current state-of-the-art methods.

To improve fusion efficiency in Multiview-based methods, Ku et al \cite{ku_joint_2018} proposed AVOD (aggregate view object detection) network including a region proposal network and a second stage detector network. A novel architecture, is introduced in the \acrshort{rpn} step, which can perform multi-modal feature fusion to produce region proposals for small objects with a high recall. 

\subsubsection{Segmentation-based Methods}
Achieving higher recall rates and outperformance in complexed scenes than Multiview-based methods, segmentation-based methods remove most background points with the semantic segmentation techniques and then generate plentiful high-quality points on foreground points. Shi et al. \cite{shi_pointrcnn_2019} come up with the PointRCNN, which generates proposals by segmenting point clouds directly into foreground objects and refines proposals by combining local spatial features and semantic features. The process is shown in Figure \(\ref{fig:LiDAR Region Proposal}\) (b),

A sequential fusion method, PointPainting was proposed by Sourabh et al.\cite{vora_pointpainting_2020}, which is a flexible framework that has three stages: semantic segmentation, fusion, and object detection. In the semantic segmentation network, pixel-wise segmentation scores are computed and served as compact image features, and then points are painted with the scores and fed into detectors. On the KITTI benchmark, PointPainting offering a fast, accurate and robust results in extensive experimentation.

\subsubsection{Frustum-based Methods}
Qi et al., the constructor of PointNet, contributed a pioneering work using frustum-based method, called F-PointNets\cite{qi_frustum_2018}. The procedure is shown in Figure \(\ref{fig:LiDAR Region Proposal}\) (c), which extracts 3D frustum proposals from 2D candidate regions generated by image object detectors firstly, then carry out instance segmentation to predict the 3D mask of the target object, and lastly perform regression using PointNet variants to estimate a modal 3D bounding box. By leveraging both 2D detectors and 3D \acrshort{dl} techniques for object localization, this model is able to provide accurate 3D bounding boxes under situations of occlusion or partial data. 

\begin{figure}[!htbp]
\centering
\includegraphics[scale=0.7]{Graphics/LiDAR Region Proposal.png}
\caption{Typical networks for three categories of region proposal-based 3D object detection methods.\cite{guo_deep_2020} From top to bottom: (a) multi-view based, (b) segmentation-based and (c) frustum-based methods.}
\label{fig:LiDAR Region Proposal}
\end{figure}

\subsubsection{Other Methods}
By deeply intergrating 3D voxel \acrshort{cnn} and PointNet-based set abstraction, a recent unified 3D object detection framework PointVoxel-RCNN was introduced by Shi et al.\cite{shi_pv-rcnn_2021}. It is a two-stage strategy, specifically voxel-to-keypoint and keypoint-to-grid. In the first step, voxel-based CNN is adopted to learn voxel-wise multiscale features and generate accurate 3D proposals, and then voxel-wise features are groupped to aggregate key point features via PointNet-based set abstraction. In the second stage,  a ROI-grid pooling module incorporated with key point features is established to learn proposal features for subsequent proposals refinement and confidence prediction. The empirical results demonstrate outstanding performance with high ranking on KITTI dataset and Waymo Open dataset\cite{sun_scalability_2020}.
Shi et al.\cite{shi_pv-rcnn_2021} proposed PointVoxel-RCNN (PV-RCNN) to leverage both 3D convolutional network and PointNet-based set abstraction for the learning of point cloud features. Specifically, the input point clouds are first voxelized and then fed into a 3D sparse convolutional network to generate high-quality proposals. The learned voxel-wise features are then encoded into a small set of key points via a voxel set abstraction module. In addition, they also proposed a keypoint-to-grid \acrshort{roi} abstraction module to capture rich context information for box refinement. Experimental results show that this method outperforms previous methods by remarkable margin. 

To improve the preliminary work PointRCNN from the inspiration of 3D ground-truth boxes, Shi et al proposed a Part-\(A^{2}\) net\cite{shi_points_2020}, which is composed of two stages: part-aware stage and part-aggregation stage. The part-aware stage utilizes accurate intra-object parts locations provided by ground truth boxes and groups the intra-object information via \acrshort{roi}-aware point cloud pooling module. Then in the part-aggregation stage, spatial relationship of the pooled intra-object info are used to refine the box location. This method introduces intra-object part locations learning for the first time and effectively enhance the 3D object detection performance.

\subsection{Single Shot Methods}
Extending the end-to-end approach in image to 3D object detection, single shot method is a single-stage network that can predict class probability and regress 3D bounding boxes directly. Zhou and Tuzel\cite{zhou_voxelnet_2017} introduce VoxelNet, which merges the feature extraction and bounding box generation into one stage. VoxelNet partitions point cloud into equally spaced voxels firstly, and transforms points in each voxel into features via \acrshort{vfe} layer, resulting in a volumetric representation of points clouds, which can cooperate with \acrshort{rpn} for further detection work. The performance of VoxelNet is strong, but the large computational cost and low speed constrain its feasibility in real-time applications. 
\subsubsection{Discretization-based Methods}
Adopting the voxel-based feature extraction method, Yan et al.\cite{graham_3d_2017} put forward a novel approach aiming to increase the speed and orientation estimation performance\cite{yan_second_2018}. Following the voxel feature extractor, a spatially sparse convolution is used to extract feature from z-axis before generating \acrshort{bev} images, and GPU-based generation algorithm are used for sparse convolution to enhance the speed further. Considering the large loss caused by the difference in orientation between ground truth and prediction when their bounding boxes are overlapping, an innovative angle loss regression is proposed to improve performance. A data augmentation approach based on capacity is introduced which exhibits substantial enhancement in convergence speed and performance.

A new encoder PointPillars, proposed by Lang et al.\cite{lang_pointpillars_2019}, transforms point clouds to pillars and utilizes PointNets to extract features on pillars. The network incorporates three stages: a encoder learn features and scatter back the learned features to a pseudo image, a backbone of 2D convolution convert the pseudo-image into a concatenation of all feature maps, and use \acrshort{ssd} methods to predict 3D bounding boxes in the last step. From the experimental results on the KITTI\cite{geiger_vision_2013} dataset, PointPillars shows dominant outperformance in AP at a higher speed, comparing with other existing methods.

\section{Adversarial Attacks and Defenses}
Evolved from back-propagation algorithm, gradient-based attacks make a slight modification to develop a perturbation vector for the input. Keeping the model weights constant and considering the input as a variable, gradients of each input element can be obtained and can be utilized to generate perturbation vector. Some widely used adversarial attack techniques include FGSM\cite{goodfellow_explaining_2015}, BIM(Iterative-FGSM\cite{kurakin_adversarial_2017}), FGSM with momentum\cite{dong_boosting_2018}, PGD\cite{madry_towards_2019} etc.

FGSM(Fast gradient sign method) ,proposed by Goodfellow et al.\cite{goodfellow_explaining_2015} is the main approach used in this paper. The general idea is to generate adversarial examples with the use of the gradient of the loss function with respect to an upper bound of the perturbation. The gradient of the loss function can be computed in deep neural networks in a fast speed, so it is named FGSM. This method is widely accepted in 2D image object detection tasks\cite{liu_detection_2018,kurakin_adversarial_2017}, also provides insight in 3D point clouds attack\cite{zeng_adversarial_2019}. More specific explanation of loss function and description of FGSM variants will be displayed in the next section.

Jacobian-based Saliency Map Attack (JSMA) is a practical gradient-based attack method relies on Jacobian matrix, is firstly proposed in image classification work by Papernot et al.\cite{papernot_limitations_2015}. It’s a forward derivative matrix used to build adversarial saliency maps, incorporating a specific perturbation in the input features to generate adversarial samples. This method achieves 97\% adversarial success rate with modification on average 4.02\% input features in each sample. JSMA is applicable to control adversaries with respect to the perturbation choice based on the adversarial goals. Some variants of JSMA grant this method more flexibility, M-JSMA lower the requirements of specifying the target class and the direction of perturbation, but still provides competitive speed and performance\cite{wiyatno_maximal_2018}, Weighted-JSMA utilizes the output probabilities to apply a weighting on the saliency map, showing the faster and more efficient non-targeted attacks than JSMA. Based on WJSMA, Taylor-JSMA take additional step to penalize external input features, giving better performance in target attack than WJSMA\cite{wiyatno_maximal_2018}.

Besides attacks in images, JSMA method can also be extended to attacking point clouds classifiers. Liu et al.\cite{liu_extending_2019} modified the JSMA slightly by considering gradients of three dimensions of the point and add perturbations in all three dimensions. Depending on the gradients, the value of dimensions can be changed, leading to the increase of candidate points to be perturbed. However, the attack effectiveness of JSMA is not satisfying with a lower success rate empirically compared with other FGSM and FGSM variants.

Defense is the operation to train neural network to make the attacks fail as much as possible. Papernot et al.\cite{papernot_distillation_2016} suggest a defensive mechanism named defensive distillation, to train DNN classifier to be more robust to perturbation. First step is to train a network on the training data with a softmax temperature, then use a probability vector which has additional class probabilities knowledge to train a distilled network at the same temperature and same training data. The empirical results on the MNIST dataset indicates resilience increase of DNNS to adversaries and improve class generalizability.

Defense on JSMA and its variants are investigated in(Probabilistic Jacobian-based Saliency Maps Attacks). By adding adversarial samples produced by JSMA, WJSMA, TJSMA to the training set, networks against TJSMA and WJSMA provide the better performance than JSMA from the view of defender.

Applied the optimization method to craft adversarial samples, Carlini and Wagner\cite{carlini_towards_2017} construct three powerful attacks \(L_{0},L_{1},L_{\infty}\), which can defeat the defensive distillation(distillation as a defense to adversarial perturbations against deep neural networks), providing a better baseline for evaluating the efficacy defenses model. They generate adversarial examples by iteratively minimize the distance metric between real and adversarial image under different constraints. CW attacks perform well in terms of success rate on IMAGENET dataset\cite{carlini_towards_2017,rottmann_detection_2021}, but it shows high computational cost and less convenient in the real-time applications\cite{combey_probabilistic_2020}. C\&W attack approach illuminates further adversarial attack approach such as EAD\cite{chen_ead_2018},\cite{sharif_accessorize_2016},()

DeepFool is another optimization-based method, proposed by Moosavi-Dezfooli et al.\cite{moosavi-dezfooli_deepfool_2016}. This method suggests an iterative linearization algorithm DeepFool, which can effectively compute adversarial perturbations and accurately evaluate the robustness of classifiers in a high speed, expecting to be a baseline method on large-scale datasets.

Xiao et al.\cite{xiao_generating_2019} were the first to utilize generative adversarial networks(GANs) to produce adversarial samples.  Applying to black-box attack and semi-white box attack, AdvGAN achieves high attack success rate and exhibits resilient under existing defenses. Besides, unrestricted adversarial examples trained on AG-GAN\cite{odena_conditional_2017} was introduced by Song et al.\cite{song_constructing_2018}. Specifically, the adversarial examples are generated from scratch using conditional generative models rather than perturbing existing points, providing an instructive and innovative view to construct adversarial examples. Simultaneously, GAN-based approaches has been attracting growing interests recently and a promising future is approaching.

Inspired by C\&W approach, Xiang et al.\cite{xiang_generating_2019} make an attempt to craft adversarial samples of 3D point clouds and propose a novel algorithm against PointNet\cite{qi_pointnet_2017} and create adversarial examples in two perspectives. One is adversarial point perturbation, by negligibly shifting existing points, another is adversarial point generation, by placing sets of scattered and independent points or some point clusters similar to the original object.After extensive experiment with ModelNet40 dataset, very high success rates (over 99\%) are demonstrated for six designed attacks under different perturbation metrics, providing a guideline for further 3D adversarial samples design. Optimization attack algorithms are vigorously promoting creative techniques in crafting 3D adversarial samples, such as in \cite{wicker_robustness_2019,athalye_synthesizing_2018}

Yang et al.\cite{yang_adversarial_2019} propose a skeleton-detach attack method on point clouds, which enhance the speed than previous gradient-based and optimization-based attacks. The method is conducted by detaching the most critical points from the critical subset referred in PointNet iteratively. Tansferability of attacks between different networks, an unsettled problem in the method of Xiang et al.\cite{xiang_generating_2019}, is successfully solved in this approach, but the success rate is declined compared with previous works.

To overcome the hindrance of speed exposed in gradient-based and optimization-based method, and flexibility limitation in skeleton detach methods\cite{yang_adversarial_2019}, Zhou et al.\cite{zhou_lg-gan_nodate} introduce a faster and flexible method label guided adversarial network(LG-GAN) to attack 3D point clouds. The general process is as follows, LG-GAN leverages an adversarial network to extract the features of input point clouds, then use the label encoder to merge specified label information into features, the last step is to feed the encoded features into the decoder to produce adversarial samples. By attacking point cloud network such as PointNet, PointNet++ and DGCNN along with other existing attack methods, Zhou et al. prove that LG-GAN outperforms FGSM and IFGSM in both speed and success rate, and reaches a similar performance as C\&W method with substantially higher speed.

Aiming at generating adversarial attacks on 3D point clouds with transferability of networks, Hamdi et al develop dubbed AdvPC\cite{vedaldi_advpc_2020}, in which the attack is generated with an accessible victim network and the adversarial sample can be applied to an inaccessible and unseen transfer network. An auto-encoder(AE), which can reconstruct the perturbed input, is introduced in the method to generate more transferable attacks. Through multiple tests on point clouds network such as PointNet, PointNet++, DGSNN, high attack success rate and better transferability than other existing 3D attacks are provided. Additionally, Hamdi et al also demonstrate that compared to other baselines on the ModelNet40, AdvPC can increase the ability to break defenses by approximately 40\%.

Denoiser and Upsampler Network(DUP-Net\cite{zhou_dup-net_2019}), proposed by Zhou et al, is a defense approach for adversarial attacks on 3D point cloud classification. This network is merge of SOR(statistical outlier removal\cite{rusu_towards_2008}) layer and PU-Net\cite{li_pu-gan_nodate}, presenting that additional robustness is offered by removing statistical outliers and the upsampler network can defend well against adversarial attacks such as \acrshort{cw} attacks.
     
% !TeX root = ../thesis.tex

\chapter{Methodology}
\label{sec:concepts}
\section{Datasets}
\subsection{KITTI Dataset}
In this paper, proposed attacks are conducted on the KITTI object detection dataset\cite{geiger_vision_2013}, which is a widely used benchmark dataset for adversarial attack research. In the KITTI 3D object detection benchmark, point clouds dataset is composed of a trainval set with 7481 training samples and a test set containing 7518 samples, involving 80256 labeled objects in total. Object class is provided in the dataset, where 'car', 'pedestrian' and 'cyclist' are the main concerned classes in the object detection studies. In addition, a leaderboard of detection method is published and continually updated on the KITTI website\cite{KITTI}, in which average precision is the main indicator to evaluate detection performance on the three main categories. Three difficulty levels "Easy, moderate, hard" are used to evaluate the detector, depending on the occlusion and truncation. More explanation of evaluation metrics will be provided in section \(\ref{evaluation Metrics}\).

\subsection{DENSE Dataset}
\textit{Autolabeling} 

Piewak et al.\cite{piewak_boosting_2018} proposed autolabeling method for labeling point clouds with classes. It uses camera and LiDAR installed with minimum instance to record the scenes simultaneously. Then uses the semantic segmentation network\cite{cordts_understanding_nodate}, which performs good at that time, trained on Cityscapes\cite{cordts_cityscapes_2016} for image segmentation. For point projection right to the reference camera plane, they proposed ego-motion correction. The general process is shown in Figure \(\ref{fig:Autolabeling}\).
\begin{figure}[!htbp]
\centering
\includegraphics[scale=0.5]{Graphics/Autolabeling.png}
\caption{Autolabeling process. Segmentation network to do semantic segmentation in images and point projection to align pixels and point clouds and finish the labeling.\cite{piewak_boosting_2018}}
\label{fig:Autolabeling}
\end{figure}
\textit{DENSE\cite{DENSE}}

DENSE is a dataset in Ulm University, which seek to focus on how adverse weather influence the autonomous driving perception system. It creates the datasets under several adverse weather like fog,snow and rain. Heizler et al.\cite{heinzler_cnn-based_2020} followed the autolabeling process to gives the each point cloud a weather class. In In Figure \(\ref{fig:DENSE Dataset}\), the result of autolabeling is shown in right graph. Fog, rain and clean class have shown in different colors. Cause DENSE dataset includes many subsets. We only used the subset from Heizler's work. The subset uses 4 chamber to record point clouds of car, pedestrian and etc. under man-made weather for training and testing. They also have a on road dataset for further testing. This subset is good to observe the specialty of point clouds under different weather and simulate point clouds under adverse weather with point clouds under clean weather.

\begin{figure}[!htbp]
\centering
\includegraphics[scale=0.5]{Graphics/DENSE.png}
\caption{DENSE Dataset}
\label{fig:DENSE Dataset}
\end{figure}
\section{Evaluation Metrics}
\label{evaluation Metrics}
\subsection{KITTI}
\subsubsection{3D Bounding Box}
In object detection task, bounding box is an imaginary rectangle to mark the spatial location of the objects of interest. In 2D image detection, a bounding box can be determined by two methods, one is two-corner representation, rendered by the coordinates \(xy\) of the upper-left corner \((x_{min},y_{max})\) and the lower-right corner \((x_{max},y_{min})\), and another commonly used method is center-width-height representation, described by the bounding center center coordinates \((x,y)\) and the box width and height. 

3D bounding box annotations, including 3D size described in length, width and height and object’s rotation and translation, are provided for each object in the KITTI dataset\cite{geiger_vision_2013}. An illustration of 3D bounding box coordinates is in Figure \(\ref{fig:3D bounding box coordinate system}\). Specifically, under the assumption that the observed objects are pinned to the ground, only yaw angle is considered in rotation, and roll and pitch angles are assumed to be zero relative to the ground plane. The coordinates of 8 vertices of the 3D bounding box can be calculated by seven parameters, namely length, width, height, object center coordinates (x,y,z) and rotation angle, realizing the parameterization of 3D bounding box. 

\begin{figure}[!htbp]
\centering
\includegraphics[scale=1]{Graphics/3D bounding box coordinate system.png}
\caption{3D bounding box coordinate system}
\floatfoot{This figure illustrates some parameters of 3D bounding box from the top view. Object coordinates are defined by the center point at the bottom of 3D bounding box, of which the coordinate system is constructed with respect to the LiDAR coordinate system. \(R_{z}\) is yaw angle representing the rotation. 3D size is determined by width(\(w\)), length(\(l\)) and height(\(h)\), which is not shown in the top view.\cite{geiger_vision_2013} }
\label{fig:3D bounding box coordinate system}
\end{figure}

\subsubsection{\acrshort{iou}}
Intersection over union(\acrshort{iou}) is an indicator showing the similarity of the ground truth bounding box and the detected bounding box, which is equal to the proportion of bounding box intersection to the union. It can be formulated as following:

\begin{equation}
          IoU_{b_{1},b_{2}} = \frac{A(b_{1}\cap b_{2})}{A(b_{1}\cup b_{2})} 
\end{equation}
Where \(A(\cdot)\) denotes the area and b means the bounding box. When \acrshort{iou} is larger than the threshold, it can be counted as positive. In the KITTI, \acrshort{iou} \(\geq\) 0.7 is required in the detection of cars, while \acrshort{iou} \(\geq\) 0.5 is accepted when recognizing pedestrians and cyclists.

\subsubsection{Average Precision}
Average precision (\acrshort{ap}) is a well-established indicator and is used to evaluate the detection performance in the KITTI. Following the PASCAL \acrshort{voc} criteria \cite{everingham_pascal_2010}, \acrshort{ap} is determined by the precision-recall curve. Precision describes how accurate the prediction is and recall measures how good you can find the positive. Specifically, precision is equal to the proportion of correct detections in all detections produced by the detector, and recall means the ratio of successful detections to all ground truth objects. The mathematical definitions are as following:
\begin{equation}
\label{precision}
          Precision = \frac{TP}{TP+FP}
\end{equation}
\begin{equation}
\label{recall}
          Recall = \frac{TP}{TP+FN} 
\end{equation}
Where \(TP\) represents true positive, implying the ground truth is present and model successfully detect the object. \(FP\) is false positive, indicating a detection when the ground truth is not present. \(FN\) means false negative, signifying the ground truth is present but not being detected by the detector.

From function \((\ref{precision})\) and \((\ref{recall})\), precision and recall are both between 0 and 1. Discretizing recall value uniformly into 11 equal-spaced points, 0,0.1,0.2,...,1. a fixed recall value set is defined. For any recall \(r^{'}\geq r\), we replace the precision value for r with maximum precision. Average precision is computed as the average of maximum precision value of the 11 recall values\cite{everingham_pascal_2010}, as function (\(\ref{AP}\)) and average precision always falls within 0 and 1.
\begin{equation}
\label{AP}
\begin{split}
    AP &= \frac{1}{11} \sum\limits_{r\in\{{0,0.1,...,1}\}} AP_{r} \\
    &=\frac{1}{11} \sum\limits_{r\in\{{0,0.1,...,1}\}} p_{interp}(r)\\
    where  \\
    &p_{interp}(r) = \max\limits_{r^{'}\geq r}p(r^{'})
\end{split}
\end{equation}
 
To promote fair comparison of the results, KITTI changed the AP definition using 40 recall positions to replace the 11 recall positions from Oct.2019\cite{KITTI,simonelli_disentangling_2019}. 
\subsubsection{Average Orientation Similarity}
\acrshort{aos}(Average Orientation Similarity) is the key 3D metric used in the KITTI dataset to evaluate the performance of detecting objects and assessing 3D orientation jointly\cite{geiger_are_2012}. Similar to the structure of \acrshort{ap}, \acrshort{aos} is the mean value of maximum normalized cosine similarity of 11 recall values\footnote{40 recall positions instead of 11 recall positions are used to calculate \acrshort{aos} after Oct.2019} and is defined as following:
\begin{equation}
    AOS = \frac{1}{11} \sum\limits_{r\in\{{0,0.1,...,1}\}}  \max\limits_{r^{'}\geq r}s(r^{'})
\end{equation}
Where \(s(\cdot)\) of a specific recall represents the orientation similarity, which is a normalized variant of the cosine similarity and is defined as:
\begin{equation}
    s(r) = \frac{1}{\abs{ D(r)}} \sum\limits_{i\in D(r)}  \frac{1+cos\Delta_{\theta}^{(i)}}{2} \delta_{i}
\end{equation}
Where \(D(r)\) is a set containing all object detections at a specific recall rate, delta describes the angle difference between estimated orientation and the ground truth orientation of the detection i. \(\delta_{i}\) is a parameter to penalize multiple detections to a single object and is determined by the overlapping of the bounding box. \(\delta_{i}\) is assigned to 1 when the detection i can be allocated to a ground truth bounding box(\acrshort{iou}\(\geq\)0.5), otherwise \(\delta_{i}\) is equal to 0.




\textit{Metric for adversarial attacks on LiDAR-based object detector}
As the simple evaluation metric in attacks\cite{xie_adversarial_2017,liu_adversarial_2019,li_robust_2019,zhang_towards_2019,wei_transferable_2019} under image object network, drop of AP of detection model could be used as the metric for effort of adversarial attacks. So we also pick the drop of AP score as our metric for evaluate the adversarial attack under LiDAR-based object detector.
\subsection{DENSE}
\section{Target LiDAR Detector Models}
\subsection{SECOND}
Sparsely Embedded Convolutional Detection (\acrshort{second}) is proposed by Yan et al. for LiDAR-based detection\cite{yan_second_2018}. \acrshort{second} detector is composed of three main components: a voxel feature extractor, a sparse convolutional layer and an \acrshort{rpn}(region proposal network). The first step is to convert raw point clouds into voxel features and coordinates, then a voxel feature extractor including two voxel feature encoding layers and a fully connected network is utilized to transfer input features into C-dimensional output features, where C is the number of output channels. Afterwards, sparse convolutional layer is used to learn z-axis information and transfer the sparse 3D information into a 2D \acrshort{bev} image. In the final phase, an \acrshort{rpn} incorporated three stage convolutions is used to predict the class, regression offsets and direction. 
\begin{figure}[!htbp]
\centering
\includegraphics[scale=1]{Graphics/SECOND.png}
\caption{\acrshort{second} detector structure}
\floatfoot{This flowchart represents the structure of \acrshort{second} detector. The input of the detector is a raw point cloud. Voxel features and coordinates converted from the raw point clouds are firstly processed by the feature extractor. Later, a sparse \acrshort{cnn} is applied to reshape 3D data into image-like 2D information and \acrshort{rpn} is used to generate prediction lastly.}
\label{fig:SECOND}
\end{figure}
\subsection{PointPillar}
PointPillars is a novel encoder introduced by Lang et al.\cite{lang_pointpillars_2019}, which can utilize only 2D convolutional layers to achieve end-to-end learning. The procedure of PointPillars is illustrated in Figure \(\ref{fig:PointPillars}\). Pillar Feature Net, Backbone and Detection Head are the three main components in PointPillars. At the beginning, LiDAR point clouds are converted to the stacked pillars tensor, and then a simplified version of PointNet\cite{qi_pointnet_2017} is created to learn features from the stacked pillars, the encoded features are scattered back to produce a 2D pseudo image to be used in the backbone, all these works are conducted in the pillar feature net. The backbone contains a 2D CNN to process features into high spatial resolution and produce a final output which is a concatenation of all features. The detection head is a single shot detector to detect and regress the 3D bounding box.
\begin{figure}[!htbp]
\centering
\includegraphics[scale=1]{Graphics/PointPillars.png}
\caption{PointPillars Network}
\floatfoot{This figure demonstrate the structure of PointPillars network. Raw point clouds are converted into a sparse pseudo image in the pillar feature net. Then the pseudo image is processed by the backbone, which is a 2D convolutional network, into a high level representation. At last the detection head is used to perform 3D detection.}
\label{fig:PointPillars}
\end{figure}
\subsection{PointRCNN}
PointRCNN is a two-stage 3D object detection approach proposed by Shi et al.\cite{shi_pointrcnn_2019}. Stage 1 is a novel strategy to generate 3D proposals in a bottom-up manner directly from point cloud, by learning point-wise features to segment the whole scene point clouds into foreground and background points. Compared with other existing methods, this proposal generation method constrains the search space and enhances the efficiency. Subsequently, a region of interest pooling operation is conducted to learn more semantic features and local spatial information of those proposals. Then the pooled points of each proposal are transformed into canonical coordinates and combined with the pooled features along with the segmentation mask to be fed into the network for further refinement, including 3D bounding box locations and foreground object confidence.
\begin{figure}[!htbp]
\centering
\includegraphics[scale=1]{Graphics/PointRCNN.png}
\caption{PointRCNN Architecture}
\floatfoot{This graph shows the structure of PointRCNN. A point cloud network is used to learn point-wise features and produce a feature vector. Then a bottom-up scheme is used to generate 3D bounding box proposals from the segmented foreground points. After performing an \acrshort{roi} pooling, the pooled points and corresponding features of each proposal are transformed into canonical coordinate for further refinement.}
\label{fig:PointRCNN}
\end{figure}
\subsection{Part-\(A^{2}\)}
Shi et al. extend their preliminary work PointRCNN to a novel network Part-\(A_{2}\)\cite{shi_points_2020}. 
Part-\(A_{2}\) comprises two stages: part-aware stage and part-aggregation stage. The task of part-aware stage is to predict intra-object part locations and learn point-wise features, and part-aggregation stage mainly contributes to part information aggregation and box refinement. An overview of the Part-A2 framework is displayed in Figure \(\ref{fig:PartA2}\). To be specific, 3D Point clouds are fed into the encoder-decoder network and transformed into feature maps. Considering different scenarios, two strategies are used to generate proposals, where anchor-based strategy can achieve high recall rates at the expenses of computation and memory cost while anchor-free strategy is more memory efficient. By fully utilizing part supervision from the ground-truth box, intra-object part location information are predicted and high-quality proposals are generated in the part-aware stage. Subsequently, via \acrshort{roi}-aware point cloud pooling module, intra-object part locations within the same proposal are grouped. With the aggregated spatial information, convolutional network is utilized to rescore the box proposals and refine the box locations. 

\begin{figure}[!htbp]
\centering
\includegraphics[scale=0.8]{Graphics/Part-A2.png}
\caption{Framework of Part-\(A_{2}\)}
\floatfoot{The left orange part illustrates the work in part-aware stage. Point clouds data are fed in to the network, which learns the point-wise features for semantic segmentation and estimation of the intra-object part locations. Two strategies anchor-free and anchor-based are used to generate 3D proposals. The right green part shows how part-aggregation stage operates. An \acrshort{roi}-aware pooling is performed to aggregating information within 3D proposals, Sparse \acrshort{cnn}s and sparse pooling are applied in the part-aggregation stage to produce accurate confidence prediction and box refinement.}
\label{fig:PartA2}
\end{figure}
\subsection{PointVoxel-RCNN}
A new contribution from Shi et al is PointVoxel-RCNN(PV-RCNN), which is an deeply intergrated network of both voxel-based and PointNet-based network\cite{shi_pv-rcnn_2021}. Two transformation steps are proposed: voxel to keypoint encoding step and keypoint to grid \acrshort{roi} feature capture step. More specific procedures are shown in Figure 8. After going through voxelization, processed point clouds are fed into 3D sparse convolutional network to learn multi-scale voxel-wise information and 3D proposals are generated using RPN. In addition, a set of keypoints are selected from the raw point clouds by \acrshort{fps} (furthest point sampling) method. A novel voxel set abstraction module is introduced to intergrate the voxel-wise features learned at convolution neural layers into selected keypoints. After the keypoint weighting module, keypoint features combined with 3D box proposals(\acrshort{roi}) are combined in the \acrshort{roi}-grid pooling module to capture proposal-specific features. Using the \acrshort{roi} feature of each proposal, a refine network is performed in order to generate box refinement and confidence prediction.
\section{Attack and Defense Methods}
\subsection{\acrshort{fgsm} and FGSM Variants}
\subsubsection{\acrshort{fgsm}}
Fast gradient sign method (\acrshort{fgsm}) is proposed by Goodfellow et al, which is a one-step attack algorithm founded on the idea to maximize loss function subject to an upper bound of the perturbation.( Explaining and harnessing adversarial examples ). Panda example is a well-known classical adversarial attack with FGSM approach, which is shown in Figure \(\ref{fig:FGSM}\).  Formally, FGSM can create an attack as following:
\begin{equation}
          x^{'} = x + \epsilon \cdot sign(\nabla_{x}J(\theta,x,y)) 
\end{equation}
Where x is the benign data point, \(x^{'}\) is the perturbed data point, \(\theta\) are parameters of the model, y represent ground truth labels, \(J(\theta,x,y)\) is loss function, \(\nabla_{x}J(\theta,x,y)\) denotes the gradient of loss function taken with respect to the input data x, and the gradients can be efficiently computed by backpropagation. The sign of the gradient is the perturbation direction. \(\epsilon\) represents the magnitude of the perturbation.
\begin{figure}[!htbp]
\centering
\includegraphics[scale=0.5]{FGSM.png}
\caption{An adversarial attack by FGSM}
\floatfoot{The adversarial sample is crafted by applying FGSM to GoogLeNet\cite{szegedy_going_2015}. By adding imperceptible perturbation, where the \(\epsilon\) is 0.007, the classifier incorrectly recognizes the panda as a gibbon with high confidence.\cite{goodfellow_explaining_2015}}
\label{fig:FGSM}
\end{figure}

\subsubsection{FGSM Variants}
\paragraph{Iterative FGSM (IFGSM) and \acrshort{pgd}} 
One drawback of FGSM is that the perturbed data points might be unbounded, which is physically prohibited in image classification problems. To solve this issue,  Kurakin et al.\cite{kurakin_adversarial_2017} introduce an improved version by iteratively apply \acrshort{fgsm} multiple times(T) with a step size of alpha, which is equal to \(\frac{\epsilon}{T}\) that denotes iterations times.
\begin{equation}
          x^{'}_{0} = x, \quad x_{t+1}^{'}=Clip^{\epsilon}_{x}\{x_{t}^{'}+\alpha \cdot sign(\nabla_{x}J(\theta,x_{t}^{'},y))\} 
\end{equation}
Empirical studies have shown that iterative FGSM show stronger performance in white-box attacks comparing with \acrshort{fgsm}, but the transferability of \acrshort{ifgsm} is worse than \acrshort{fgsm}\cite{kurakinAdversarialMachineLearning2017,tramer_ensemble_2020}.

\acrshort{pgd}(projected gradient descent) applies similar method as basic \acrshort{ifgsm}, but starts with a randomly initialized perturbation and projections. A coordinate-wise random step size is utilized, indicating that each entry of the gradient is scaled by a factor which is uniformly and randomly chosen\cite{madry_towards_2019}.
\paragraph{\acrshort{mifgsm}(Momentum Iterative Fast Gradient Sign Method)}
Dong et al. propose an \acrshort{mifgsm} by intergating the \acrshort{ifgsm} and momentum, in order to escape from poor local maxima and stabilize update directions\cite{dong_boosting_2018}. 
 \begin{equation}
 \begin{split}
          & g_{t+1} = \mu \cdot g_{t}+\frac{J(\theta,x^{'}_{t},y)}{\|\nabla_{x}J(\theta,X_{t}^{'},y\|_{1}}\\
          &x_{t+1}^{'}=Clip^{\epsilon}_{x}\{x_{t}^{'}+\alpha \cdot sign(\nabla_{x}J(\theta,x_{t}^{'},y))\} 
\end{split}
\end{equation}
Where \(\mu\) is the decay factor of the momentum term and \(g_{n}\) is the accumulated gradient at iteration \(t\).

\subsubsection{Norms}

The norm \(\|x - x_{0}\| \)is a measure of distance between x and \(x_{0}\). When generating adversarial examples, the bound is restricted according to the following norms:

\(L_{\infty}\)-norm:\(\Phi(x) =\|x \minus x_{0}\|_{\infty} \), which minimizes the maximum element of the perturbation.

\(L_{1}\)-norm:\(\Phi(x) =\|x \minus x_{0}\|_{1} \),  a convex surrogate of the \(L_{0}\)-norm.



Additionally, L2.5 norm, which is the normalized L2 norm and proposed by Liu et al.\cite{liu_extending_2019}, is also used in the paper.
 \begin{figure}[!htbp]
\centering
\includegraphics[scale=0.5]{Graphics/Norms.png}
\caption{Illustrations of norms.}
\label{fig:Norm}
\end{figure}
 
 
 
 
 
 
 
 S in FGSM is sign, which represents  \(L_{\infty}\) norm.
 Moosavi-Dezfooli et al.\cite{moosavi-dezfooli_deepfool_2016} proposed DeepFool to acquire the closest distance from the original input to the decision boundary of adversarial examples, which uses \(L_{2}\) norm to replace sign norm for a better result in image.
 \begin{center}
          \(x^{*} =x+\sigma\frac{\nabla_{x}J(x,y)}{\|\nabla_{x}J(x,y)\|_{2}} \)
\end{center}
 It could also be increased more broad norm, \(L_{\rho}\in [0,\infty)\):
 
 \(L_{1}\) norm as Manhattan metric to replace sign norm. 
 
 Liu et.al(\cite{liu_adversarial_2019}) proposed a new version of \(L_{2}\) norm named \(L_{2.5}\) norm, which is normalized \(L_{2}\) norm's result. 
\textit{Perturbation Selection and Adjustment} 
The attacker can use the network sensitivity information about the input differences to evaluate the dimensionality of the misclassification of the target that is most likely to cause the least interference. There are two types of perturbed input dimensions:

 \textbf{Perturb all the input dimensions}: Goodfellow et al.\cite{goodfellow_explaining_2015} proposed a method to perturb each input dimension, but there is a small amount of perturbation in the direction of the gradient sign calculated using the FGSM method. This method effectively minimizes the Euclidean distance between the original sample and the corresponding adversarial sample. 

\textbf{Perturb a selected input dimension}: Papernot et al.\cite{papernot_limitations_2015} choose to focus more on the complex process involving saliency mapping to perturbation by selecting only a limited number of input dimensions. The purpose of using a saliency map is to assign a value to a combination of input dimensions, which indicates whether the combination will contribute to adversarial goal if it is disturbed. This method effectively reduces the number of input features that are disturbed when making adversarial examples. In order to select the input dimensions that constitute the disturbance, all dimensions are sorted in descending order of adversarial saliency value. A legal example of the target class t is the saliency value  \(S(x,t)[i]\) of the component i of x, which is evaluated using the following equation:
\begin{equation}
S(x,t)[i]=\left\{
             \begin{array}{lr}
             0, & if \frac{\partial F_{t}}{\partial x_{i}}(x)\textless 0 \quad or \quad \sum_{j\neq t} \frac{\partial F_{t}}{\partial x_{i}}(x) \textgreater 0 \\
             \frac{\partial F_{t}}{\partial x_{i}}(x)\begin{vmatrix}
                \sum_{j\neq t} \frac{\partial F_{t}}{\partial x_{i}}(x)
                \end{vmatrix}, & otherwise\\
            
             \end{array}
\right.
\nonumber
\end{equation}
where \([\frac{\partial F_{t}}{\partial x_{i}}]_{ij}\) could be not difficultly derived from the Jacobian Matrix \(J_{F}\) of the victim model F. Input dimensions are added to perturbation \(\delta_{x}\) in the reducing order of the saliency values \(S(x,t)[i]\) until the acquiring image \(x_{*} = x + \delta x\) is misclassified by the network F.

The two perturbation selection types have their own assets and disadvantages. The first one is fine matched for the quick crafting of many adversarial examples rather with a larger perturbation, so it's more simple to tell the differences. The second one decreases the perturbations of relatively higher computation cost.

A better explanation of how LiDAR works, which is dealing with LiDAR signals in spherical coordinates \((r,\tau,\theta)\). In Figure \(\ref{fig:Lidar Rangeimage}\), Li et al.\cite{li_lidar_2020} used Velodyne UltraPuck to explain the principle of LiDAR. The vertical angle elevation for each laser beam is settled and the azimuth angle is determined by the scanning time and motor speed. Hence every range reading can be expressed by \(P_{i,j}=(\rho_{i,j},\tau_{i,j},\theta_{i,j}\), where i represents a certain laser beam and j is azimuth-angle index. Range readings fill into a predefined data buffer to complete range image. According to the azimuth and elevation of each point, it's easier to transform range image to point clouds.
  \begin{figure}[!htbp]
\centering
\includegraphics[scale=0.45]{Graphics/LIDAR RangeImage.png}
\caption{The range view of spinning lidar (Velodyne UltraPuck) for further processing. The range image (32 × 1,800) in pseudocolor facilitates the subsequent processing\cite{li_lidar_2020}}
\label{fig:Lidar Rangeimage}
\end{figure}
 
 One LiDAR point's azimuth and elevation is fixed. After perturbation to this point, LiDAR's physical characteristic should not be changed. As Figure \(\ref{fig:Adjustment}\) have shown one point named with "Original Point" is perturbated. When the perturbed point is green point, it have the same two angles as the original point. If perturbed point goes to red point, then the perturbed point's two angles should be adjusted to the original degrees, in order for keeping the physical character of LiDAR point clouds.
 \begin{figure}[!htbp]
\centering
\includegraphics[scale=0.5]{Graphics/Adjustment.png}
\caption{LiDAR Point Coordinate System}
\label{fig:Adjustment}
\end{figure}

\paragraph{Defense Mechanism}
In this paper, defense strategy chosen to against the adversarial attacks is adversarial training, which is intuitive and widely accepted method.() The main idea is to inject \acrshort{fgsm}-generated advasarial samples into the training set in a manner that allows the trained model classify the adversarial examples correctly\cite{goodfellow_explaining_2015}. Formally, the loss function can be formulated as below:

\begin{equation}
          \widetilde{J}(\theta,x,y) = c J(\theta,x,y)+(1-c)J(\theta,x+\epsilon sign(\nabla_{x}J(\theta,x,y)),y)
\end{equation}
Where \(x\) represents the benign sample, \(c\) is a hyper parameter set to balance the accuracy of benign and adversarial samples, 


The prime propose of the adversarial training is to make the victim network more robust by adding adversarial samples to the training dataset. Adversarial training is a common brute force algorithm where the defense model easily produces lots of adversarial examples. The adversarial training is an argumentation process by injecting the adversarial examples. The training formular is given by \cite{goodfellow_explaining_2015}  

where J is initial loss function. The core intention behind this approach is to increase the robustness of the victim network by securing that it will classify the same category for precise as the perturbed examples. 

\subsubsection{Applications}
DNN is firstly developed in images. And FGSM as an efficient adversarial attack way is proposed firstly in neuron network in image classification. With the development of DNN, people is not satisfied only with image classification, there's more works and more networks are proposed to solve image segmentation and image detection for precise represent and complex scene. FGSM is then transferred to newer networks.
\paragraph{Image Semantic Segmentation}
Arnab et al.\cite{arnab_robustness_nodate} also evaluated FGSM and Iterative FGSM based adversarial attacks for semantic segmentation networks. Their work shows FGSM and its variants is also efficient in many semantic segmentation networks and pointed out that many attacks for classification do not straight transfer to segmentation mission. Cause the gradient can easily get from segmentation networks and FGSM just need the gradient to complete the attack. 
\paragraph{Image Object Detection}
Zhang et al.\cite{zhang_towards_2019} performs PGD(FGSM variant) as adversarial training method to defense object detectors, which can significantly improve the robustness and can defense adversarial examples generated by DAG\cite{xie_adversarial_2017} and RAP\cite{li_robust_2019}.

Liu et al.\cite{liu_mi-fgsm_2020} integrate MI-FGSM into target detection initially and achieve adequate attack performance, which attack method obtain higher success rate and more efficient and powerful than the attack algorithm such as PGD (Projected Gradient Descent) on object detection. They also show adversarial examples on Faster R-CNN\cite{ren_faster_2016} not only lead to misclassification, but there is also a wrong positioning, which is the wrong candidate boundary box. Those adversarial examples have established remarkable effects on black-box attack and white-box attack.

\paragraph{Point Cloud Segmentation}
PointNet and PointNet++ shows their elegant architecture and impressive results in point cloud classification mission. Liu et al.\cite{liu_extending_2019} transfer firstly FGSM attack and defenses to PointNet and PointNet++. They transfered with untargeted FGSM, targeted FGSM, FGSM's variants IFGSM, MIFGSM and PGD with 4 different gradient norms including \(L_{2.5}\) proposed by themselves. And also show the adversarial training is useful for defencing the adversarial attacks from FGSM and its variants.
\paragraph{FGSM in Point Cloud Object Detection}
With rapid development of LiDAR-based object detectors, the adversarial attacks on these networks is tiny. We plan to transfer FGSM and FGSM variants with 4 different fast gradient norms to attack and defense the detection model.
For perturbation selection part, different from images' 3 dimensions, LiDAR point clouds have 4 dimensions which are 4 features: \(x\) coordinate, \(y\) coordinate, \(z\) coordinate and intensity \(r\). So there are two different kinds of attack types: one is only attack one feature, which is separately attack \(x\) coordinate, \(y\) coordinate, \(z\) coordinate and intensity \(r\). The other is attack multiple features: \(xyz\) coordinates, and the combination of \(xyz\) coordinates and intensity \(r\). 


\subsection{Point Dropping}

\subsubsection{Critical Point Definition}

Different from multi-view networks and volumetric models, as the pioneer network as point-based method in point cloud classification shows their fast inference speed and capability of keeping more space information. PointNet\cite{qi_pointnet_2017} proved symmetric function could achieve permutation invariance to tackle the unordered problem of point clouds. They uses max-pooling layer as symmetric function to get points as global features. The gained points are called critical points.

Zheng et al.\cite{zheng_pointcloud_2019} proposed an adversarial method to drop the critical point compare with randomly point drop. The method of removing critical point shows its effectiveness with relative high sucess rate.
PointNet as a pioneer point-based network. Many subsequent networks pick it as the baseline. PointPillars\cite{lang_pointpillars_2019} is one of them. Although PointPillars uses the simplified PointNet, it can't get critical point by max-pooling layer and cause there is not max-pooling layer, but max layer in PointPillars.

\subsubsection{Critical Features in PointPillars}
 
PointNet shows the significant performance directly dealing with point clouds, which gives hug influences to subsequent point clouds based networks. PointPillars\cite{lang_pointpillars_2019} is one of them, which uses simplified PointNet\cite{qi_pointnet_2017} called Pillar Feature Net(Figure \(\ref{fig:Pillar Feature Net}\) have shown part of Pillar Feature Net) as one part of their work. Pillar Feature Net's first two step is import for definition of critical features. The first step is to stack points into pillars, the second step is to get learned features in each pillar. The learned features is defined by us as critical features.

\begin{figure}[!htbp]
\centering
\includegraphics[scale=1]{Graphics/Pillar Feature Net.png}
\caption{Part of Pillar Feature Net\cite{lang_pointpillars_2019}}
\label{fig:Pillar Feature Net}
\end{figure}

\subsubsection{Definition of Critical Points in PointPillars}
If we want to drop critical point in PointPillars as Zheng\cite{zheng_pointcloud_2019} did in PointNet, we need to define critical points with critical features. As we know the process of PointPillars, it will get critical features per pillar. So the critical point we defined will also occur in each pillar. In OpenPCDet setting of PointPillars, there are 32 points per pillar and 64 features before max layer, after max layer through points dimension, there's 64 critical features per pillar left. We proposed two criterion for the definition of critical point based on this process:

1. Critical point in each pillar is the point with maximum critical feature value: After we get 64 critical features, we could obtain 1 maximal critical feature among these 64 critical features. The point, which contain the maximum critical feature value , corresponds to our criterion 1.

2. Critical point is the point that has most critical features. In operation of max layer, we've got 64 features. When we store the indexes of the maximal values, then we know how much critical features each point per pillar includes. We specify the point, which contains most number of critical features, as critical point based on our criterion 2.

In Figure \(\ref{fig:Critical Points}\), cells with critical features are pinpointed in colored background. In the whole table, 10 is the maximal critical feature value, and it's located at the third row, so this point is a critical point from first criterion. When counting the number of critical features, the fourth row has four colored cells, showing it has most features than other points. So, this point is a critical point based on second criterion.
\begin{figure}[!htbp]
\centering
\includegraphics[scale=0.5]{Graphics/Critical Points.png}
\caption{Pseudo Process of Catching Critical Features}
\label{fig:Critical Points}
\end{figure}

\subsection{Common Perturbation Generation}
Hendrycks et al.\cite{hendrycks_benchmarking_2019} summarise the image processing methods and divided into image corruption and image perturbation. And build the ImageNet-C benchmark datasets and ImageNet-P benchmark datasets with these methods. There's 15 different corruption methods including Gaussian noise, shot noise, weather transfer, several blur methods, compression methods and etc. Each methods gives 5 levels to create 75 kinds of datasets. The perturbation methods contain motion blur, zoom blur, snow, brightness, translate, tilt(viewpoint variantion through 3D rotations), rotate and scale perturbations.

Inspired of Hendrycks, Michaelis et al.\cite{michaelis_benchmarking_2020} transfered this kind of work to autonomous driving datasets\cite{cordts_cityscapes_2016} rather than ImageNet\cite{deng_imagenet_2009}. As for creating methods, they selected only weather transfer methods for establishing "truly" datasets.
\subsubsection{Methods}
The weather transfer have been well developed in images. In images
 1. Add Gaussian noise perturbation.

2. Drop points randomly.

3. Add points surround cluster and ground points. Use RANSAC to segement ground points of lidar point clouds and then get clusters with DBSCAN. Random pick the some points in cluster and ground points to do the gaussian noise perturbation and then add the perturbated points to the original points as the addition of noise points.
\subsubsection{Simulation of adverse weather}
adverse weather is point clouds, which LiDAR acquired under adverse weather, including rain, snow, fog, etc. From Figure \(\ref{fig:Point Clouds Differences}\) we could seen in the point clouds under adverse weather are significant different with clean point clouds. 
\begin{figure}[!htbp]
    \centering
    \subfigure[Clean Point Cloud\cite{geiger_vision_2013}]{
        \includegraphics[width = \textwidth / 2 ]{Graphics/Clean Point Cloud.png}
        \label{fig:Clean Point Cloud}
    }
    \hspace{10pt}
     %add desired spacing between images, e. g. ~, \quad, \qquad, \hfill etc.
     %(or a blank line to force the subfigure onto a new line)
    \subfigure[Snow Point Cloud\cite{pitropov_canadian_2021}]{
        \includegraphics[width = \textwidth / 2 ]{Graphics/Snow Point Cloud.png}
        \label{fig:Snow Point Cloud}
    }
    \caption[Short Description for List of Figures]{The Differences between Clean Point Cloud and Snow Point Cloud}
    \label{fig:Point Clouds Differences}
\end{figure}

\textit{Creating methods in Point Clouds}: Considering establishing "truly" point clouds based on available point clouds. Gaussian noise could be occured in the process of LiDAR drawing point clouds\cite{gatt_micro-doppler_2000}. Merging with perturbation selection in FGSM we could get following method 1. LiDAR works like bat, which needs a emitter to emit laser beams, after some beams reflected by surface of objects, the receiver could received some of the reflected beam lasers. So the receiver could get nothing, which represents there's no points. Randomly drop points method means the dropped point cannot be received by the receiver. Weather transfer work is still not available in point cloud. And there's less creating methods than image domain.

1. Points perturbation with Gaussian noise applied with different features: intensity \(r\), \(xyz\) coordinates, and the combination of \(xyz\) coordinates and intensity \(r\).

2. Points randomly drop: in order to cause same influence to point cloud, the number of dropping points is getting from specific percentage of the total number of points.

\textit{Metric using benchmark datasets}
 To evaluate the robustness, rPP, which is the relative performance under point cloud perturbation, is used. rPP is the ratio of mPP and \(P_{clean}\):
\begin{center}
          \(rPP = \frac{mPP}{AP_{clean}} \)
\end{center}
\(P_{clean}\) is the average precision of the detection model under clean dataset, and mPP is the mean value of average precision of detection model under benchmark datasets:
\begin{center}
          \(mPC = \frac{1}{N_{C}}\sum_{c=1}^{N_{C}}\sum_{s=1}^{N_{S}}{AP_{C,S}} \)
\end{center}
\({P_{C,S}}\): performance evaluated on data changed with changes C with perturbation and drop points under severity level S

\({N_{C}}\): number of changes, including different methods

\({N_{S}}\): severity levels



\textit{Simulate Methods}

 
  \textit{Evaluation Metric: Weather Classification} 
 
 Vargas et.al\cite{vargas_rivero_weather_2020} have shown, use simple statistic parameters of point clouds: standard deviation and mean value of \(x\) coordinates and mean and standard deviation of echo-pulse width, which is intensity, etc. 13 parameters as input for a weather classifier.
 \textit{Evaluation Metric: Weather Points Semantic Segmentation Network}
 
 
 Evaluate the performance of simulation, weather point semantic segementation network is introduced. As the figure 6 shown, the DENSE\cite{DENSE} Dataset labeled each point cloud with a weather category, which is adverse weather consists of rain and fog. Heinzler et.al \cite{heinzler_cnn-based_2020} proposed WeatherNet training on DENSE Dataset. His results is shown in Table \(\ref{tab:DENSE}\)
 \begin{table}[h!]
  \begin{center}

    \begin{tabular}{|c|c|c|c|} % <-- Alignments: 1st column left, 2nd middle and 3rd right, with vertical lines in between
      \hline
      \multirow{2}{*}{Methods} & \multicolumn{3}{c|}{Weather Class}\\
    \cline{2-4}
       &\textbf{Clean(mIOU)} & \textbf{Rain(mIOU)}& \textbf{Fog(mIOU)}\\
      \hline
      WeatherNet & 93.35 & 90.92&88.81\\
      Cylinder3D & 98.91 & 94.61&91.63\\
      \color{blue}
      Delta & \color{blue}+5.56 & \color{blue}+3.69&\color{blue}+2.28\\
    \hline
    \end{tabular}
    \caption{Training Results on DENSE Dataset}
  \end{center}
    \label{tab:DENSE}
\end{table}
Cylinder3D\cite{zhu_cylindrical_2020} performs well in Semantic KITTI\cite{behley_semantickitti_2019}, which is an object semantic datasets. 


% !TeX root = ../thesis.tex

\chapter{Implementation}
\label{sec:implementation}
We use PointPillars\cite{lang_pointpillars_2019} as the detection model at most Experiments. Cause PointPillars' inference speed is fast and has relative higher precision. After adversarial attacks and defenses apply to victim network. The processing speed will be 2 times or 3 times slower than original one. So the running time of victim model is more important. The code is based on OpenPCDet\cite{openpcdet2020}, We use R40\cite{simonelli_disentangling_2019} as Average Precision(AP) to evaluate different levels of cars. The threshold of IoU for different levels of cars is 0.7.
\section{FGSM}
\subsection{Selection of Epsilons}

\(\epsilon\) is a parameter in FGSM, which determines the extent of perturbation. In our experiment, we choose epsilons ranging from 0.001 to 0.2 to do attacks, and the attack result is in Figure \(\ref{fig:Select Epsilons}\). By observing the curve, we choose the epsilons in each tuning point to take further experiments: 0.01, 0.02, 0.04. 

As the Figure \(\ref{fig:FGSM Comparison}\) shown, we apply 3 different \(\epsilon\) on the point clouds and we could see the changes between original point clouds(OPC) and OPC added perturbation calculated by FGSM with \(\epsilon\) 0.04 is easy to tell. OPC is almost the same as OPC added perturbation by FGSM with \(epsilon\) 0.01. For  

\begin{figure}[htbp]
\centering
\subfigure[OPC]{
\begin{minipage}[t]{0.25\linewidth}
\centering
\includegraphics[width=1in]{Graphics/LidarCar0.png}
%\caption{fig1}
\end{minipage}%
}%
\subfigure[FGSM with \(\epsilon\) 0.01]{
\begin{minipage}[t]{0.25\linewidth}
\centering
\includegraphics[width=1in]{Graphics/LidarCar0.01.png}
%\caption{fig2}
\end{minipage}%
}%
\subfigure[FGSM with \(\epsilon\) 0.02]{
\begin{minipage}[t]{0.25\linewidth}
\centering
\includegraphics[width=1in]{Graphics/LidarCar0.02.png}
%\caption{fig2}
\end{minipage}
}%
\subfigure[FGSM with \(\epsilon\) 0.04]{
\begin{minipage}[t]{0.25\linewidth}
\centering
\includegraphics[width=1in]{Graphics/LidarCar0.04.png}
%\caption{fig2}
\end{minipage}
}%
\centering
\caption{FGSM with different \(\epsilon\) on Point Clouds, where OPC represents original point clouds and the other three figures is the results after applying FGSM attack with different \(\epsilon\) on original point clouds.}
\label{fig:FGSM Comparison}
\end{figure}

\begin{figure}[!htbp]
\centering
\includegraphics[scale=0.9]{Graphics/Select Epsilons.png}
\caption{Results of FGSM Attack with different Epsilons}
\label{fig:Select Epsilons}
\end{figure}

\subsection{Attack Features}
In Chapter 3 we introduce the features represents different dimensions of LiDAR. In Figure 4.2, when attacking only one feature, we observe x coordinate is the most sensitive. When attacking multiple features, the combination of coordinates \(xyz\) and intensity \(r\) has the most significant performance.
\begin{figure}[!htbp]
\centering
\includegraphics[scale=0.9]{Graphics/Attack Features.png}
\caption{Performance of FGSM Attack on Different Features}
\label{fig:Attack Features}
\end{figure}

% PC Comparison
\begin{figure}[htbp]
\centering

\subfigure[OPC]{
\begin{minipage}[t]{0.25\linewidth}
\centering
\includegraphics[width=1in]{Graphics/LidarCar0.png}
%\caption{fig1}
\end{minipage}%
}%
\subfigure[Coordinates \(xyz\) with \(\epsilon\) 0.01]{
\begin{minipage}[t]{0.25\linewidth}
\centering
\includegraphics[width=1in]{Graphics/LidarCar0.01.png}
%\caption{fig2}
\end{minipage}%
}%
\subfigure[Coordinates \(xyz\) with \(\epsilon\) 0.02]{
\begin{minipage}[t]{0.25\linewidth}
\centering
\includegraphics[width=1in]{Graphics/LidarCar0.02.png}
%\caption{fig1}
\end{minipage}%
}%
\subfigure[Coordinates \(xyz\) with \(\epsilon\) 0.04]{
\begin{minipage}[t]{0.25\linewidth}
\centering
\includegraphics[width=1in]{Graphics/LidarCar0.04.png}
%\caption{fig2}
\end{minipage}%
}%

\subfigure[OPC]{
\begin{minipage}[t]{0.25\linewidth}
\centering
\includegraphics[width=1in]{Graphics/LidarCar0.png}
%\caption{fig2}
\end{minipage}
}%
\subfigure[Coordinate \(x\) with \(\epsilon\) 0.01]{
\begin{minipage}[t]{0.25\linewidth}
\centering
\includegraphics[width=1in]{Graphics/lidar_x_0.01.png}
%\caption{fig2}
\end{minipage}
}%
\subfigure[Coordinate \(x\) with \(\epsilon\) 0.02]{
\begin{minipage}[t]{0.25\linewidth}
\centering
\includegraphics[width=1in]{Graphics/lidar_x_0.02.png}
%\caption{fig2}
\end{minipage}
}%
\subfigure[Coordinate \(x\) with \(\epsilon\) 0.04]{
\begin{minipage}[t]{0.25\linewidth}
\centering
\includegraphics[width=1in]{Graphics/lidar_x_0.04.png}
%\caption{fig2}
\end{minipage}
}%

\subfigure[OPC]{
\begin{minipage}[t]{0.25\linewidth}
\centering
\includegraphics[width=1in]{Graphics/LidarCar0.png}
%\caption{fig2}
\end{minipage}
}%
\subfigure[Coordinate \(y\) with \(\epsilon\) 0.01]{
\begin{minipage}[t]{0.25\linewidth}
\centering
\includegraphics[width=1in]{Graphics/lidar_y_0.01.png}
%\caption{fig2}
\end{minipage}
}%
\subfigure[Coordinate \(y\) with \(\epsilon\) 0.02]{
\begin{minipage}[t]{0.25\linewidth}
\centering
\includegraphics[width=1in]{Graphics/lidar_y_0.02.png}
%\caption{fig2}
\end{minipage}
}%
\subfigure[Coordinate \(y\) with \(\epsilon\) 0.04]{
\begin{minipage}[t]{0.25\linewidth}
\centering
\includegraphics[width=1in]{Graphics/lidar_y_0.04.png}
%\caption{fig2}
\end{minipage}
}%

\subfigure[OPC]{
\begin{minipage}[t]{0.25\linewidth}
\centering
\includegraphics[width=1in]{Graphics/LidarCar0.png}
%\caption{fig2}
\end{minipage}
}%
\subfigure[Coordinate \(z\) with \(\epsilon\) 0.01]{
\begin{minipage}[t]{0.25\linewidth}
\centering
\includegraphics[width=1in]{Graphics/lidar_z_0.01.png}
%\caption{fig2}
\end{minipage}
}%
\subfigure[Coordinate \(z\) with \(\epsilon\) 0.02]{
\begin{minipage}[t]{0.25\linewidth}
\centering
\includegraphics[width=1in]{Graphics/lidar_z_0.02.png}
%\caption{fig2}
\end{minipage}
}%
\subfigure[Coordinate \(z\) with \(\epsilon\) 0.04]{
\begin{minipage}[t]{0.25\linewidth}
\centering
\includegraphics[width=1in]{Graphics/lidar_z_0.04.png}
%\caption{fig2}
\end{minipage}
}%

\centering
\caption{Point Clouds Comparision under 4 Different Features. Where OPC represents original point clouds. the others represents the results after applying different features attack with different \(\epsilon\) on original point clouds. Cause the point cloud visualization with intensity \(r\) is hard to tell, so we only choose 4 features about coordinates: coordinate \(x\), coordinate \(y\), coordinate \(z\), coordinates \(xyz\). }
\label{fig:PC under 4 Features Comparsion}
\end{figure}
\subsection{Adversarial Training}

We use the traditional FGSM adversarial defense method to do the adversarial training. We've introduced, there's other 3 norms (\(L_{1}\),\(L_{2}\),\(L_{2.5}\) to replace \(L_{\infty}\) in FGSM. So we do the adversarial training with 4 different norms and \(\epsilon\) 0.02. And then we got the evaluate result shown in Table \(\ref{tab:FGSM Adversarial Training}\) compare with original model. We've got around 30 models with different evaluation results for each model. We pick the best model based on AP of R40 of 3D bounding boxes of different levels of cars.

FGSM-based Adversarial training is an efficient way to defense adversarial attacks of FGSM. The loss of adversarial training of FGSM:
\begin{center}
          \(Loss_{train} = 1/2*(Loss_{original}+Loss_{perturbation}) \)
\end{center}
 \begin{table}[!htbp]
  \begin{center}
    
    \begin{tabular}{|c|c|c|c|} % <-- Alignments: 1st column left, 2nd middle and 3rd right, with vertical lines in between
      \hline
      \multirow{2}{*}{Models} & \multicolumn{3}{c|}{Average Precision of Different levels of Car}\\
    \cline{2-4}
       &\textbf{Moderate(AP)} & \textbf{(Easy(AP)}& \textbf{Hard(AP)}\\
      \hline
      Original Model & 78.50 & 89.03 & 75.67\\
      \(L_{1}\) Defense Model &  77.83 & 88.26& 74.81\\
      \(L_{2}\) Defense Model & 78.09 & 89.07&75.12\\
      \(L_{2.5}\) Defense Model & 77.75 & 87.41&74.90\\
      \(L_{\infty}\) Defense Model & 76.88 & 87.17&74.22\\
      IFGSM \(L_{2}\) Defense Model & 77.85 & 87.75&75.00\\
    \hline
    \end{tabular}
\caption{Adversarial Training Results Compare with Original Results}
  \end{center}
  \label{tab:FGSM Adversarial Training}
\end{table}

We've found the adversarial defense will not make the model performs better, but could achieve similar 3D AP Scores. But whether the defense model can defense the adversarial attack? We've implemented FGSM attacks with 4 different norm on 5 different models (1 original model and 4 norms defensed models).

\begin{figure}[!htbp]
\centering
\includegraphics[scale=0.6]{Graphics/Defense Models Attack.png}
\caption{Performance of Attack to Defensed Models}
\label{fig:Defensed Models}
\end{figure}

We've seen the Figure 4.3, the 5 colors of lines represents the AP score from different models. Here are some results we've got from this graph:

1. The attack results of the original model is this blue line. In all these four pictures, this blue line is substantial below other lines, so we conclude FGSM defense models can defend FGSM attack.

2. In the \(L_{\infty}\) attack, the average precision of these five models are smaller than that of other three attacks. So,we know \(L_{\infty}\) attack has the most substantial impact on these five models.

3. In the \(L_{\infty}\) attack, the purple line, which is \(L_{\infty}\) defensed model, is the highest, showing that \(L_{\infty}\) defensed model performs best in the \(L_{\infty}\) attack.

4. The green line is the highest in the these three pictures, indicating that \(L_{2}\) defensed model 
performs best in \(L_{1}\), \(L_{2}\), \(L_{2.5}\) attack. In general, \(L_{2}\) norm defensed model has the best performance.

Considering all these results, we use \(L_{\infty}\) attack and \(L_{2}\) defensed model for the 
next step.

We've also plot the comparison of 4 norms attack under 3 different epsilons on original model, as Figure \(\ref{fig:PC under 4 Noms Comparsion}\) shown. 

% PC Comparison
\begin{figure}[htbp]
\centering

\subfigure[OPC]{
\begin{minipage}[t]{0.25\linewidth}
\centering
\includegraphics[width=1in]{Graphics/LidarCar0.png}
\end{minipage}%
}%
\subfigure[\(L_{\infty}\) with \(\epsilon\) 0.01]{
\begin{minipage}[t]{0.25\linewidth}
\centering
\includegraphics[width=1in]{Graphics/LidarCar0.01.png}
\end{minipage}%
}%
\subfigure[\(L_{\infty}\) with \(\epsilon\) 0.02]{
\begin{minipage}[t]{0.25\linewidth}
\centering
\includegraphics[width=1in]{Graphics/LidarCar0.02.png}
\end{minipage}%
}%
\subfigure[\(L_{\infty}\) with \(\epsilon\) 0.04]{
\begin{minipage}[t]{0.25\linewidth}
\centering
\includegraphics[width=1in]{Graphics/LidarCar0.04.png}
\end{minipage}%
}%

\subfigure[OPC]{
\begin{minipage}[t]{0.25\linewidth}
\centering
\includegraphics[width=1in]{Graphics/LidarCar0.png}
\end{minipage}
}%
\subfigure[\(L_{1}\) with \(\epsilon\) 0.01]{
\begin{minipage}[t]{0.25\linewidth}
\centering
\includegraphics[width=1in]{Graphics/Lidar_1_0.01.png}
\end{minipage}
}%
\subfigure[\(L_{1}\) with \(\epsilon\) 0.02]{
\begin{minipage}[t]{0.25\linewidth}
\centering
\includegraphics[width=1in]{Graphics/Lidar_1_0.02.png}
\end{minipage}
}%
\subfigure[\(L_{1}\) with \(\epsilon\) 0.04]{
\begin{minipage}[t]{0.25\linewidth}
\centering
\includegraphics[width=1in]{Graphics/Lidar_1_0.04.png}
\end{minipage}
}%

\subfigure[OPC]{
\begin{minipage}[t]{0.25\linewidth}
\centering
\includegraphics[width=1in]{Graphics/LidarCar0.png}
\end{minipage}
}%
\subfigure[\(L_{2}\) with \(\epsilon\) 0.01]{
\begin{minipage}[t]{0.25\linewidth}
\centering
\includegraphics[width=1in]{Graphics/Lidar_2_0.01.png}
\end{minipage}
}%
\subfigure[\(L_{2}\) with \(\epsilon\) 0.02]{
\begin{minipage}[t]{0.25\linewidth}
\centering
\includegraphics[width=1in]{Graphics/Lidar_2_0.02.png}
\end{minipage}
}%
\subfigure[\(L_{2}\) with \(\epsilon\) 0.04]{
\begin{minipage}[t]{0.25\linewidth}
\centering
\includegraphics[width=1in]{Graphics/Lidar_2_0.04.png}
\end{minipage}
}%

\subfigure[OPC]{
\begin{minipage}[t]{0.25\linewidth}
\centering
\includegraphics[width=1in]{Graphics/LidarCar0.png}
\end{minipage}
}%
\subfigure[\(L_{2.5}\) with \(\epsilon\) 0.01]{
\begin{minipage}[t]{0.25\linewidth}
\centering
\includegraphics[width=1in]{Graphics/Lidar_2.5_0.01.png}
\end{minipage}
}%
\subfigure[\(L_{\infty}\) with \(\epsilon\) 0.02]{
\begin{minipage}[t]{0.25\linewidth}
\centering
\includegraphics[width=1in]{Graphics/Lidar_2.5_0.02.png}
\end{minipage}
}%
\subfigure[\(L_{2.5}\) with \(\epsilon\) 0.04]{
\begin{minipage}[t]{0.25\linewidth}
\centering
\includegraphics[width=1in]{Graphics/Lidar_2.5_0.04.png}
\end{minipage}
}%

\centering
\caption{Point Clouds Comparision under 4 Norms}
\floatfoot{Where OPC represents original point clouds. the others represents the results after applying FGSM attack with different norm type and \(\epsilon\) on original point clouds }
\label{fig:PC under 4 Noms Comparsion}
\end{figure}


\subsection{FGSM Variants Attack and Defense}
To check whether FGSM defensed model can defense FGSM variants attack, we compare two 
models, the original model, which is non-defensed, and \(L_{2}\) defensed model. The attack 
norm we use here is \(L_{\infty}\). And the results are Figure 4.4.
\begin{figure}[!htbp]
\centering
\includegraphics[scale=0.5]{Graphics/FGSM Variants.png}
\caption{Model performance under different attacks}
\label{fig:FGSM Variants}
\end{figure}
% PC Comparison
\begin{figure}[htbp]
\centering

\subfigure[OPC]{
\begin{minipage}[t]{0.25\linewidth}
\centering
\includegraphics[width=1in]{Graphics/LidarCar0.png}
\end{minipage}%
}%
\subfigure[\(L_{\infty}\) with \(\epsilon\) 0.01]{
\begin{minipage}[t]{0.25\linewidth}
\centering
\includegraphics[width=1in]{Graphics/LidarCar0.01.png}
\end{minipage}%
}%
\subfigure[\(L_{\infty}\) with \(\epsilon\) 0.02]{
\begin{minipage}[t]{0.25\linewidth}
\centering
\includegraphics[width=1in]{Graphics/LidarCar0.02.png}
\end{minipage}%
}%
\subfigure[\(L_{\infty}\) with \(\epsilon\) 0.04]{
\begin{minipage}[t]{0.25\linewidth}
\centering
\includegraphics[width=1in]{Graphics/LidarCar0.04.png}
\end{minipage}%
}%

\subfigure[OPC]{
\begin{minipage}[t]{0.25\linewidth}
\centering
\includegraphics[width=1in]{Graphics/LidarCar0.png}
\end{minipage}
}%
\subfigure[IFGSM with \(\epsilon\) 0.01]{
\begin{minipage}[t]{0.25\linewidth}
\centering
\includegraphics[width=1in]{Graphics/lidar_ifgsm_0.01.png}
\end{minipage}
}%
\subfigure[IFGSM with \(\epsilon\) 0.02]{
\begin{minipage}[t]{0.25\linewidth}
\centering
\includegraphics[width=1in]{Graphics/lidar_ifgsm_0.02.png}
\end{minipage}
}%
\subfigure[IFGSM with \(\epsilon\) 0.04]{
\begin{minipage}[t]{0.25\linewidth}
\centering
\includegraphics[width=1in]{Graphics/lidar_ifgsm_0.04.png}
\end{minipage}
}%

\subfigure[OPC]{
\begin{minipage}[t]{0.25\linewidth}
\centering
\includegraphics[width=1in]{Graphics/LidarCar0.png}
\end{minipage}
}%
\subfigure[PGD with \(\epsilon\) 0.01]{
\begin{minipage}[t]{0.25\linewidth}
\centering
\includegraphics[width=1in]{Graphics/lidar_pgd_0.01.png}
\end{minipage}
}%
\subfigure[PGD with \(\epsilon\) 0.02]{
\begin{minipage}[t]{0.25\linewidth}
\centering
\includegraphics[width=1in]{Graphics/lidar_pgd_0.02.png}
\end{minipage}
}%
\subfigure[PGD with \(\epsilon\) 0.04]{
\begin{minipage}[t]{0.25\linewidth}
\centering
\includegraphics[width=1in]{Graphics/lidar_pgd_0.04.png}
\end{minipage}
}%

\subfigure[OPC]{
\begin{minipage}[t]{0.25\linewidth}
\centering
\includegraphics[width=1in]{Graphics/LidarCar0.png}
\end{minipage}
}%
\subfigure[MIFGSM with \(\epsilon\) 0.01]{
\begin{minipage}[t]{0.25\linewidth}
\centering
\includegraphics[width=1in]{Graphics/lidar_momentum_0.01.png}
\end{minipage}
}%
\subfigure[MIFGSM with \(\epsilon\) 0.02]{
\begin{minipage}[t]{0.25\linewidth}
\centering
\includegraphics[width=1in]{Graphics/lidar_momentum_0.02.png}
\end{minipage}
}%
\subfigure[MIFGSM with \(\epsilon\) 0.04]{
\begin{minipage}[t]{0.25\linewidth}
\centering
\includegraphics[width=1in]{Graphics/lidar_momentum_0.04.png}
\end{minipage}
}%

\centering
\caption{Point Clouds Comparision under 4 FGSM Variants}
\floatfoot{Where OPC represents original point clouds. the others represents the results after applying FGSM attack with different norm type and \(\epsilon\) on original point clouds }
\label{fig:PC under 4 Noms Comparsion}
\end{figure}
As the average precision under each epsilon in the right graph are larger than the left picture, 
which indicates that the defensed model effectively defense FGSM variants. In both graphs, the performance of three FGSM variants are very similar, and the blue line of FGSM attack is distinctly above those lines of its variants, so we conclude FGSM variants perform better than FGSM attack.
\section{Drop Critical Points}
\subsection{Drop Critical Points Attack}
\paragraph{Implementation}
The method we've introduced in concepts part. We've proposed two methods to define critical points with critical features. PointPillars\cite{lang_pointpillars_2019} firstly create pillars using fixed number of points per pillar with point clouds. That means it will insert zero points when the pillar can not contain enough points. When we do the experiments, we've found the point with the most critical features will mostly come from the last virtual point and the point with maximum critical feature will part of come from virtual point. We've made some adjustment to let it only remove the truly point from pillar. And we will remove points with pillars number every iteration. And we drop random point per pillar for the comparison. The drop iterations is 1 to 5.
\begin{figure}[!htbp]
\centering
\includegraphics[scale=0.3]{Graphics/Points Drop.png}
\caption{Performance of Drop Points}
\label{fig:Drop Points}
\end{figure}
\paragraph{Results}
% PC Comparison
\begin{figure}[htbp]
\centering

\subfigure[OPC]{
\begin{minipage}[t]{0.25\linewidth}
\centering
\includegraphics[width=1in]{Graphics/LidarCar0.png}
\end{minipage}%
}%

\subfigure[Drop Critical Point 1 with Iteration 1]{
\begin{minipage}[t]{0.25\linewidth}
\centering
\includegraphics[width=1in]{Graphics/lidar_criticalmaximum_1.png}
\end{minipage}%
}%
\subfigure[Drop Critical Point 2 with Iteration 1]{
\begin{minipage}[t]{0.25\linewidth}
\centering
\includegraphics[width=1in]{Graphics/lidar_criticalmost_1.png}
\end{minipage}%
}%
\subfigure[Drop Random Point with Iteration 1]{
\begin{minipage}[t]{0.25\linewidth}
\centering
\includegraphics[width=1in]{Graphics/lidar_random_1.png}
\end{minipage}%
}%

\subfigure[Drop Critical Point 1 with Iteration 2]{
\begin{minipage}[t]{0.25\linewidth}
\centering
\includegraphics[width=1in]{Graphics/lidar_criticalmaximum_2.png}
\end{minipage}%
}%
\subfigure[Drop Critical Point 2 with Iteration 2]{
\begin{minipage}[t]{0.25\linewidth}
\centering
\includegraphics[width=1in]{Graphics/lidar_criticalmost_2.png}
\end{minipage}%
}%
\subfigure[Drop Random Point with Iteration 2]{
\begin{minipage}[t]{0.25\linewidth}
\centering
\includegraphics[width=1in]{Graphics/lidar_random_2.png}
\end{minipage}%
}%

\subfigure[Drop Critical Point 1 with Iteration 3]{
\begin{minipage}[t]{0.25\linewidth}
\centering
\includegraphics[width=1in]{Graphics/lidar_criticalmaximum_3.png}
\end{minipage}%
}%
\subfigure[Drop Critical Point 2 with Iteration 3]{
\begin{minipage}[t]{0.25\linewidth}
\centering
\includegraphics[width=1in]{Graphics/lidar_criticalmost_3.png}
\end{minipage}%
}%
\subfigure[Drop Random Point with Iteration 3]{
\begin{minipage}[t]{0.25\linewidth}
\centering
\includegraphics[width=1in]{Graphics/lidar_random_3.png}
\end{minipage}%
}%

\subfigure[Drop Critical Point 1 with Iteration 4]{
\begin{minipage}[t]{0.25\linewidth}
\centering
\includegraphics[width=1in]{Graphics/lidar_criticalmaximum_4.png}
\end{minipage}%
}%
\subfigure[Drop Critical Point 2 with Iteration 4]{
\begin{minipage}[t]{0.25\linewidth}
\centering
\includegraphics[width=1in]{Graphics/lidar_criticalmost_4.png}
\end{minipage}%
}%
\subfigure[Drop Random Point with Iteration 4]{
\begin{minipage}[t]{0.25\linewidth}
\centering
\includegraphics[width=1in]{Graphics/lidar_random_4.png}
\end{minipage}%
}%

\subfigure[Drop Critical Point 1 with Iteration 5]{
\begin{minipage}[t]{0.25\linewidth}
\centering
\includegraphics[width=1in]{Graphics/lidar_criticalmaximum_5.png}
\end{minipage}%
}%
\subfigure[Drop Critical Point 2 with Iteration 5]{
\begin{minipage}[t]{0.25\linewidth}
\centering
\includegraphics[width=1in]{Graphics/lidar_criticalmost_5.png}
\end{minipage}%
}%
\subfigure[Drop Random Point with Iteration 5]{
\begin{minipage}[t]{0.25\linewidth}
\centering
\includegraphics[width=1in]{Graphics/lidar_random_5.png}
\end{minipage}%
}%

\centering
\caption{Point Clouds Comparision under 3 Drop Methods}
\floatfoot{Where OPC represents original point clouds. the others represents the results after applying FGSM attack with different norm type and \(\epsilon\) on original point clouds }
\label{fig:PC under 4 Noms Comparsion}
\end{figure}
In the Table 4.4, the green and blue columns are attacks of dropping critical points, and the purple column shows the attack by dropping points randomly. 

(1)The orange column is the highest when dropping different point numbers, so we can say the attack of critical points drop performs more effectively than random points drop.

(2)Since the green column and blue column have roughly similar height, we could say the two methods have similar results. 

\subsection{Drop Critical Points Defense}
Similar to FGSM-based adversarial defense we use the same loss ratio for normal train loss and adversarial loss. But the adversarial loss will get from the point clouds after dropping critical points. But we've tried drop pillar number points for training. Then it will make the moderate AP score drop nearly 15. After training drop less critical points, points of 5\% pillars number to points of 1\% pillars number, which didn't make AP Score performs better. We've also tried with random point drop to defense. And the random drop have two methods:
randomly drop point from each pillar and drop point randomly from point clouds. The comparison result between 4 defensed models and original model is in Table \(\ref{tab:Drop Adversarial Training}\)
 \begin{table}[!htbp]
  \begin{center}
    
    \begin{tabular}{|c|c|c|c|} % <-- Alignments: 1st column left, 2nd middle and 3rd right, with vertical lines in between
      \hline
      \multirow{2}{*}{Models} & \multicolumn{3}{c|}{AP of Different Levels of Car}\\
    \cline{2-4}
       &\textbf{Moderate(AP)} & \textbf{(Easy(AP)}& \textbf{Hard(AP)}\\
      \hline
      Original Model & 78.50 & 89.03 & 75.67\\
      Drop Critical Points Defensed Model 1 & 64.69 & 74.96& 63.11\\
      Drop Critical Points Defensed Model 2 & 65.07 & 75.37& 62.68\\
      Randomly Drop Points Defensed Model 1& 62.71 & 73.05&60.92\\
      Randomly Drop Points Defensed Model 2& 68.52 & 79.51&65.22\\
    \hline
    \end{tabular}
\caption{Adversarial Training Results Compare with Original Results. All defensed Model will drop points with 1\% of pillars numbers, Drop critical points defensed model 1 represents drop critical points with criterion 1, which is the point with most critical features. Drop critical points defensed model 2 represents drop critical points with criterion 2, which is the point with maximum critical value. Randomly drop points defensed model drop the point from pillars and randomly drop points defensed model 2 drop the point from original point.}
  \end{center}
  \label{tab:Drop Adversarial Training}
\end{table}
\subsection{Attack on Defensed Models}
We want to see if defensed models can defense critical drop attacks. We've tested 3 drop methods on 4 defensed models. But the AP scores of trained models is far from original models. So we changed the evaluation metric to relative drop of each models
\section{Points Perturbation and Points Drop}
\subsection{Test Models Robustness}
The last two sections, we've proposed adversarial attack methods based on models. We've also tried to design some adversarial methods not relying on specified model. Then we introduce the Gaussian noise perturbation and drop points randomly to create a benchmark datasets to test more models' robustness.
We've tried to create 20 datasets:

1. We add Gaussian noise perturbation with \(\mu\) 0 \(\sigma\) 0.05, 0.1, 0.15, 0.2, 0.25 to coordinate \(xyz\) of point clouds, then we do the adjustment after perturbation.

2. We add Gaussian noise with \(\mu\) 0 \(\sigma\) 0.15, 0.2, 0.25, 0.25, 0.3 to intensity \(r\) of point clouds. The intensity \(r\) of point clouds in KITTI\cite{geiger_vision_2013} datasets changes from 0 to 0.99. After perturbation of intensity \(r\), we will truncate the values exceed the range. We've found if the \(\sigma\) is smaller than 0.15, the AP score of detection model will not change.

3. We add Gaussian noise to coordinate \(xyz\) and intensity \(r\) with \(\mu\) 0 \(\sigma\) 0.05, 0.1, 0.15, 0.2, 0.25, and will do the adjustment combining method 1 and 2. 

4. We drop random points with the different percent 5\% 10\% 15\% 20\% 25\% of the total number of point clouds. We've found if we drop more than 30\% points of the total number, PointRCNN\cite{shi_pointrcnn_2019} will not work.

We've used the pretrained model in OpenPCDet and there's pretrained model of PointPillar\cite{lang_pointpillars_2019}, PointRCNN, PartA2\cite{shi_points_2020}, Second\cite{yan_second_2018}, PVRCNN\cite{shi_pv-rcnn_2021} trained under KITTI. We've used relative AP Score intorduced in Chapter 3 to test these models' robustness.
\begin{figure}[!htbp]
\centering
\includegraphics[scale=0.3]{Graphics/Relative AP.png}
\caption{Robustness of Detection Models}
\label{fig:Relative AP}
\end{figure}

In Figure 4.6, the vertical axis is the relative performance under perturbation, the highest column is PointRCNN, so we know it performs best. The performance of other detection models are 
almost the same.

The changes of \(\sigma\) equals to 0.25 to point cloud is actually huge, there's 68\% of perturbation less than 0.25, but 27\% of perturbation between 0.25 to 0.5 and 0.5 means 0.5 meter perturbation to a point. The perturbation is clearly. If we control the \(\sigma\) changes from 0.02 to 0.1 in each methods. Then we've got the relative performances are almost the same as Figure 4.7 shown.
\begin{figure}[!htbp]
\centering
\includegraphics[scale=0.4]{Graphics/Relative AP1.png}
\caption{Robustness of Detection Models}
\label{fig:Relative AP1}
\end{figure}

\subsection{Adversarial Training of Benchmark Datasets}
The datasets is not changed with the different parameters of model. So we've stored the 20 benchmark datasets and merge 20 datasets and original one into a huge dataset for training. PointPillars as the highest training speed in OpenPCDet is trained under the compound datasets to see if the benchmark datasets help detection model performs better. 

 \begin{table}[!htbp]
  \begin{center}
    
    \begin{tabular}{|c|c|c|c|} % <-- Alignments: 1st column left, 2nd middle and 3rd right, with vertical lines in between
      \hline
      \multirow{2}{*}{Models} & \multicolumn{3}{c|}{AP of Different Levels of Car}\\
    \cline{2-4}
       &\textbf{Moderate(AP)} & \textbf{(Easy(AP)}& \textbf{Hard(AP)}\\
      \hline
      Original Model & 78.50 & 89.03 & 75.67\\
      Defensed Model & 75.89 & 87.54&  73.01\\
    \hline
    \end{tabular}
\caption{PointPillars Adversarial Training on Benchmark Datasets and Original KITTI}
  \end{center}
  \label{tab:Benchmark Adversarial Training}
\end{table}
From the comparison of the results of models. Defense Model performs clearly not good as original model.
For the robustness of defensed model of PointPillars. It shows
 \begin{table}[!htbp]
  \begin{center}
    
    \begin{tabular}{|c|c|c|} % <-- Alignments: 1st column left, 2nd middle and 3rd right, with vertical lines in between
      \hline
      \multirow{2}{*}{Models} &\multicolumn{2}{c|}{Robustness of Different Datasets}\\
    \cline{2-3}
      &\textbf{Datasets1(rPP)}&\textbf{Datasets2(rPP)}\\
      \hline
      Original Model&0.76&0.93\\
      Defensed Model&0.88&0.93\\
    \hline
    \end{tabular}
\caption{Robustness on Benchmark Datasets of 2 PointPillars Models}
  \end{center}
  \label{tab:robustness pointpillar}
\end{table}
\section{Adverse Weather Point Clouds Simulation}
The objective is using point clouds under clean weather to simulate point clouds under adverse weather.

We've proposed three simulating methods:

1. Add Gaussian noise perturbation to point clouds under clean weather to simulate.

2. Drop point clouds under clean weather randomly to simulate.

3. Use RANSAC\cite{fischler_random_1981} to get ground plane point clouds cluster, and use DBSCAN\cite{rehman_dbscan_2014} to get the other point clouds parts' clusters as different objects(cars, trees, pedestrian, etc.). We randomly select some points in each cluster and add Gaussian noise perturbation to these points and add these points to original points.

It's hard for naked eyes to tell, which point clouds belong to which weather. We've found there's more precise way, weather point segment network to replace weather classification network. With a better semantic segmentation network Cylinder3D\cite{zhu_cylindrical_2020} trained on DENSE\cite{DENSE}, which performs better than WeatherNet (Table \(\ref{tab:DENSE}\)). So we use Cylinder3D as our evluation network.

 \begin{table}[h!]
  \begin{center}

    \begin{tabular}{|c|c|c|} % <-- Alignments: 1st column left, 2nd middle and 3rd right, with vertical lines in between
      \hline
      \multirow{2}{*}{Methods} & \multicolumn{2}{c|}{Weather Class}\\
    \cline{2-3}
       &\textbf{Clean Points Number} & \textbf{Rain Points Number)}\\
      \hline
      clean Point Clouds(cPC) & 9360 & 11\\
      \hline
      cPC + \(\sigma\) 0.01 perturbation & 6577 & 194\\
      \hline
      cPC + \(\sigma\) 0.05 perturbation & 7600 & 1069\\
      \hline
      cPC + \(\sigma\) 0.1 perturbation & 8142 & 2013\\
      \hline
      cPC + \(\sigma\) 0.2 perturbation & 7857 & 2455\\
      \hline
      cPC + \(\sigma\) 0.3 perturbation & 5659 & 5545\\
      \hline
      Rain Point Cloud & 4129 & 3624\\
    \hline
    \end{tabular}
    \caption{Add Gaussian Noise Perturbation Results}
  \end{center}
    \label{tab:Gaussian Noise Simulate}
\end{table}
In Table \(\ref{tab:Gaussian Noise Simulate}\), we have one clean point clouds(cPC), which aims to simulate rain point cloud with Gaussian noise perturbation. We use Cylinder3D to get the different number of rain and clean point clouds. If the ratio of rain and clean point clouds of the simulated point clouds and rain point clouds are the same, we assume we made a success simulation under evaluation. In the table we've seen when \(\sigma\) equals to 0.3, the ratio is almost the same. Then we think the simulation is success.
\include{Main/5_Conclusion_Future_Work}      

%% --------------------
%% |   Appendix+Verzeichnisse   |
%% --------------------
\appendix
\chapter{Appendix}
\label{appendix}

\section{Acronyms and Notations}
\label{sec:eval_param}

\newpage
%--------------------------------------------------------------------------------


% List of Figures
\cleardoublepage
\addcontentsline{toc}{chapter}{\listfigurename}
\listoffigures

% Liste of Tables
\cleardoublepage
\addcontentsline{toc}{chapter}{\listtablename}
\listoftables

% Bibliography
% \bibliographystyle{IEEEtranSA}   		% IEEE Alphanumeric
\bibliographystyle{ieeetr}
\cleardoublepage
\addcontentsline{toc}{chapter}{Bibliography}
\bibliography{references,thesis_bibtex}

% Online References
\ifuseglossaries
	\printglossary[type=onlineref, style=mylist]
\else
\fi

\end{document}
